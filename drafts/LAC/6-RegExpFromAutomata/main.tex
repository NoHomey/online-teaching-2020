\documentclass[12pt]{article}
\usepackage{tikz}
\usetikzlibrary{automata, positioning, arrows}
\tikzset{
->, >=stealth, node distance=2.5cm,
every state/.style={thick},
initial text=$ $
}
\usepackage[utf8]{inputenc}
\usepackage[T2A]{fontenc}
\usepackage[english,bulgarian]{babel}
\usepackage{amsmath}
\usepackage{amssymb}
\usepackage{amsthm}
\usepackage{relsize}
\usepackage{euler}

\title{Строене на регулярен израз по тотален КДА}
\author{Иво Стратев}

\begin{document}

\maketitle

\section{Конгругентност}

Нека \(\Sigma\) е крайна азбука. Въвеждаме релация между регулярните изрази над \(\Sigma\).
\[r \approx_\Sigma s \iff \mathcal{RL}_\Sigma(r) = \mathcal{RL}_\Sigma(s)\]

В сила са 

\begin{align*}
    (\forall r, s \in RegExp(\Sigma))(r \approx_\Sigma s \implies r^* \approx_\Sigma s^*) \\\\
    (\forall r, s, t, p \in RegExp(\Sigma))(r \approx_\Sigma s \;\&\; t \approx_\Sigma p \implies r . t \approx_\Sigma s . p) \\\\
    (\forall r, s, t, p \in RegExp(\Sigma))(r \approx_\Sigma s \;\&\; t \approx_\Sigma p \implies r + t \approx_\Sigma s + p)
\end{align*}
\begin{align*}
    (\forall r, s \in RegExp(\Sigma))(r \approx_\Sigma s \implies r + s \approx_\Sigma s + r) \\\\
    (\forall r, s, t \in RegExp(\Sigma))((r + s).t \approx_\Sigma (r.t) + (s.t)) \\\\
    (\forall r, s, t \in RegExp(\Sigma))(t.(r + s) \approx_\Sigma (t.r) + (t.s))
\end{align*}
\begin{align*}
    (\forall r \in RegExp(\Sigma))(\emptyset + r \approx_\Sigma r) \\\\
    (\forall r \in RegExp(\Sigma))(\emptyset . r \approx_\Sigma \emptyset) \\\\
    (\forall r \in RegExp(\Sigma))(r . \emptyset \approx_\Sigma \emptyset) \\\\
    (\forall r \in RegExp(\Sigma))(\varepsilon . r \approx_\Sigma r) \\\\
    (\forall r \in RegExp(\Sigma))(r . \varepsilon \approx_\Sigma r) \\\\
    (\forall r \in RegExp(\Sigma))((r^*)^*\approx_\Sigma r) \\\\
    (\forall r \in RegExp(\Sigma))((\varepsilon + r)^* \approx_\Sigma r) \\\\
    (\forall r \in RegExp(\Sigma))(r^* . (\varepsilon + r) \approx_\Sigma r^*) \\\\
    (\forall r \in RegExp(\Sigma))((\varepsilon + r) . r^* \approx_\Sigma r^*)
\end{align*}

\section{Теорема на Клини}
Теоремата на Клини гласи, че за всеки автоматен език същесвува регулярен израз, който описва езика.
Тоест класът на автоматните езици се включва в класът на регулярните езици.

Доказателството на теоремата дава алгоритъм за намиране на регулярен израз описващ език на даден креан тотален детерминиран автомат.

Нека \(n \in \mathbb{N}_+\) и нека \(Q = \{0, 1, \dots, n - 1\}\).
Нека \(\mathcal{A} = \langle \Sigma, Q, s, \delta, F \rangle\) е КТДА.
Тогава можем да намерим регулярен ираз \(r_{ij}^k\) съотвестващ на езика \(R_{ij}^k\), където \(i, j \in Q\) и \(k \in \{0, 1, \dots n\}\),
който описва думите, за които изичислението по автомата започнато от състояние \(i\) завършва в състояние \(j\) и преминава през междинни състояния само от множеството \(\{0, 1, \dots, k - 1\}\).

Като
\[R_{ii}^0 = \{\varepsilon\} \cup \{u \in \Sigma \;\mid\; i = \delta(i, u) \}\]
\[i \neq j \implies R_{ij}^0 = \{u \in \Sigma \;\mid\; j = \delta(i, u) \}\]
\[R_{ij}^{k + 1} = R_{ij}^k \cup R_{ik}^k \cdot (R_{kk}^k)^* \cdot R_{kj}^k\]

Така \(\mathcal{L}(\mathcal{A}) = \displaystyle\bigcup_{j \in F} R_{0j}^n\).


\section{Пример}

Нека намерим регулярен израз описващ езика на автомата.

\begin{center}
\begin{tikzpicture}
    \node[state, initial] (q0) {$0$};
    \node[state, accepting, right of=q0] (q1) {$1$};
    \node[state, right of=q1] (q2) {$2$};
    \draw (q0) edge[loop above] node{a} (q0);
    \draw (q0) edge[above] node{b} (q1);
    \draw (q1) edge[loop above] node{b} (q1);
    \draw (q1) edge[above] node{a} (q2);
    \draw (q2) edge[loop above] node{$a, b$} (q2);
\end{tikzpicture}
\end{center}

Правим следната таблица, съотвестваща на базовите случаи.
\[\begin{array}{c|ccc}
k = 0 & 0 & 1   & 2     \\ \hline
0   & \varepsilon + a & b & \emptyset \\
1   & \emptyset & \varepsilon + b & a \\
2   & \emptyset & \emptyset & \varepsilon + a + b 
\end{array}\]

След като използваме рекуретната връзка \[R_{ij}^{k + 1} = R_{ij}^k \cup R_{ik}^k \cdot (R_{kk}^k)^* \cdot R_{kj}^k\]

Получаваме следната връзка \(r_{ij}^{1} \approx r_{ij}^0 + r_{i0}^0 \cdot (r_{00}^0)^* \cdot r_{0j}^0\), където \((r_{00}^0)^* \approx (\varepsilon + a)^* \approx a^*\). Или така получаваме връзката \(r_{ij}^{1} \approx r_{ij}^0 + r_{i0}^0 \cdot a^* \cdot r_{0j}^0\). Пресмятаме за \(k = 1\).

\[r_{00}^{1} \approx r_{00}^0 + r_{00}^0 \cdot a^* \cdot r_{00}^0 \approx (\varepsilon + a) + (\varepsilon + a) \cdot a^* \cdot (\varepsilon + a) \approx \varepsilon + a + (\varepsilon + a) \cdot a^* \approx \varepsilon + a + a^* \approx a^* \]
\[r_{01}^{1} \approx r_{01}^0 + r_{00}^0 \cdot a^* \cdot r_{01}^0 \approx b + (\varepsilon + a) \cdot a^* \cdot b \approx b + a^* \cdot b \approx
\varepsilon.b + a^* \cdot b \approx (\varepsilon + a^*) \cdot b \approx a^* \cdot b\]
\[r_{02}^{1} \approx r_{02}^0 + r_{00}^0 \cdot a^* \cdot r_{02}^0 \approx \emptyset + (\varepsilon + a) \cdot a^* \cdot \emptyset \approx \emptyset + \emptyset \approx \emptyset\]
\[r_{10}^{1} \approx r_{10}^0 + r_{10}^0 \cdot a^* \cdot r_{00}^0 \approx \emptyset + \emptyset \cdot a^* \cdot (\varepsilon + a) \approx \emptyset + \emptyset \approx \emptyset\]
\[r_{11}^{1} \approx r_{11}^0 + r_{10}^0 \cdot a^* \cdot r_{01}^0 \approx
\varepsilon + b + \emptyset \cdot a^* \cdot b \approx \varepsilon + b + \emptyset \approx \varepsilon + b\]
\[r_{12}^{1} \approx r_{12}^0 + r_{10}^0 \cdot a^* \cdot r_{02}^0 \approx
a + \emptyset \cdot a^* \cdot \emptyset \approx a + \emptyset \approx a\]
\[r_{20}^{1} \approx r_{20}^0 + r_{20}^0 \cdot a^* \cdot r_{00}^0 \approx
\emptyset + \emptyset \cdot a^* \cdot (\varepsilon + a) \approx \emptyset + \emptyset \approx \emptyset\]
\[r_{21}^{1} \approx r_{21}^0 + r_{20}^0 \cdot a^* \cdot r_{01}^0 \approx
\emptyset + \emptyset \cdot  a^* \cdot b \approx \emptyset + \emptyset \approx \emptyset\]
\[r_{22}^{1} \approx r_{22}^0 + r_{20}^0 \cdot a^* \cdot r_{02}^0 \approx
\varepsilon + a + b  + \emptyset \cdot  a^* \cdot \emptyset \approx \varepsilon + a + b\]

Така получаваме таблицата
\[\begin{array}{c|ccc}
k = 1 & 0 & 1   & 2     \\ \hline
0   & a^* & a^*.b & \emptyset \\
1   & \emptyset & \varepsilon + b & a \\
2   & \emptyset & \emptyset & \varepsilon + a + b 
\end{array}\]

От рекуретната връзка получаваме връзката \[r_{ij}^{2} \approx r_{ij}^{1 + 1} \approx r_{ij}^1 + r_{i1}^1 \cdot (r_{11}^1)^* \cdot r_{1j}^1 \approx_{ij}^1 + r_{i1}^1 \cdot b^* \cdot r_{1j}^1\]

Пресмятаме за \(k = 2\).

\[r_{00}^{2} \approx r_{00}^1 + r_{01}^1 \cdot b^* \cdot r_{10}^1 \approx
a^* + a^* \cdot b \cdot b^* \cdot \emptyset \approx a^*\]
\[r_{01}^{2} \approx r_{01}^1 + r_{01}^1 \cdot b^* \cdot r_{11}^1 \approx
a^*\cdot b^* + a^* \cdot b \cdot b^* \cdot (\varepsilon + b) \approx a^*\cdot b + a^* \cdot b \cdot b^* \approx a^* \cdot b \cdot (\varepsilon + b^*) \approx a^* \cdot b \cdot b^*\]
\[r_{02}^{2} \approx r_{02}^1 + r_{01}^1 \cdot b^* \cdot r_{12}^1 \approx
\emptyset + a^* \cdot b \cdot b^* \cdot a \approx a^* \cdot b \cdot b^* \cdot a\]
\[r_{10}^{2} \approx r_{10}^1 + r_{11}^1 \cdot b^* \cdot r_{10}^1 \approx
\emptyset + (\varepsilon + b) \cdot b^* \cdot \emptyset \approx \emptyset\]
\[r_{11}^{2} \approx r_{11}^1 + r_{11}^1 \cdot b^* \cdot r_{11}^1 \approx (\varepsilon + b) + (\varepsilon + b) \cdot b^* + (\varepsilon + b) \approx (\varepsilon + b) + (\varepsilon + b) \cdot b^* \approx (\varepsilon + b) + b^* \approx b^*\]
\[r_{12}^{2} \approx r_{12}^1 + r_{11}^1 \cdot b^* \cdot r_{12}^1 \approx
a + (\varepsilon + b) \cdot b^* \cdot a
\approx a + b^* \cdot a \approx (\varepsilon + b^*) \cdot a \approx b^* \cdot a\]
\[r_{20}^{2} \approx r_{20}^1 + r_{21}^1 \cdot b^* \cdot r_{10}^1 \approx
\emptyset + \emptyset \cdot b^* \cdot \emptyset \approx \emptyset\]
\[r_{21}^{2} \approx r_{21}^1 + r_{21}^1 \cdot b^* \cdot r_{11}^1 \approx
\emptyset + \emptyset \cdot b^* \cdot (\varepsilon + b) \approx \emptyset\]
\[r_{22}^{2} \approx r_{22}^1 + r_{21}^1 \cdot b^* \cdot r_{12}^1 \approx
(\varepsilon + a + b) + \emptyset \cdot b^* \cdot a \approx \varepsilon + a + b\]

Така получаваме таблицата
\[\begin{array}{c|ccc}
k = 2 & 0 & 1   & 2     \\ \hline
0   & a^* & a^* \cdot b \cdot b^* & a^* \cdot b \cdot b^* \cdot a \\
1   & \emptyset & b^* & b^* \cdot a \\
2   & \emptyset & \emptyset & \varepsilon + a + b 
\end{array}\]

Понеже автоматът е 
\begin{center}
\begin{tikzpicture}
    \node[state, initial] (q0) {$0$};
    \node[state, accepting, right of=q0] (q1) {$1$};
    \node[state, right of=q1] (q2) {$2$};
    \draw (q0) edge[loop above] node{a} (q0);
    \draw (q0) edge[above] node{b} (q1);
    \draw (q1) edge[loop above] node{b} (q1);
    \draw (q1) edge[above] node{a} (q2);
    \draw (q2) edge[loop above] node{$a, b$} (q2);
\end{tikzpicture}
\end{center}
Той има едно финално състояние и значи \(\mathcal{L}(\mathcal{A}) = R_{01}^3 = \mathcal{RL}_{\{a, b\}}(r_{01}^3)\). Следователно трябва да пресметнем само \(r_{01}^3\).

\[r_{01}^{3} \approx r_{01}^2 + r_{02}^1 \cdot (r_{22}^2)^* \cdot r_{21}^2 \approx a^* \cdot b \cdot b^* + r_{02}^1 \cdot (r_{22}^2)^* \cdot \emptyset \approx a^* \cdot b \cdot b^* + \emptyset \approx  a^* \cdot b \cdot b^*\]

Следователно \(\mathcal{L}(\mathcal{A}) =  \{a\}^* \cdot \{b\}^+\) и търсеният регулярен израз е \(a^* \cdot b \cdot b^*\).

\end{document}
