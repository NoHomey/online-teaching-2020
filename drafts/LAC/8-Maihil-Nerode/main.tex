\documentclass[12pt]{article}
\usepackage[utf8]{inputenc}
\usepackage[T2A]{fontenc}
\usepackage[english,bulgarian]{babel}
\usepackage{amsmath}
\usepackage{amssymb}
\usepackage{amsthm}
\usepackage{euler}

\title{Релация на Майхил-Нероуд}
\author{Иво Стратев}

\begin{document}

\maketitle

\section{Въведение}
Нека \(L\) е език над \(\Sigma\), тоест \(L \subseteq \Sigma^*\).
Релацията на Майхил-Нероуд за езика \(L\) бележим с \(\approx_L\) и тя е бинарна релация над \(\Sigma^*\).
По дефиниция
\[\alpha \approx_L \beta \overset{def}{\iff} (\forall \gamma \in \Sigma^*)(\alpha.\gamma \in L \iff \beta.\gamma \in L)\]

От лекции знаем, че релацията \(\approx_L\) е релация на еквивалетнст и ако индексът ѝ е краен, то \(L\) е регулярен, защото тогава имаме конструкция за КТДА, който е и минимален.
Индексът на релацията \(\approx_L\) е мощността на множеството от класовете на еквивалетност на релацията.

\vspace*{5mm}

\par Множеството \(\Sigma^*\) е изброимо безкрайно понеже \(\Sigma\) е крайно и непразно.
Така, че индексът на \(\approx_L\) e или креан или изброимо безкраен.
Значи ако покажем изброймо безкрайно подмножество на \(\Sigma^*\), в което никой две думи не са в релация, то ще покажем и че \(L\) не е регулярен, защото тогава \(\approx_L\) няма креан индекс.

\section{Пример 1}

Нека \(L = \{c^n.a^k.b^s \; \mid\; n \in \mathbb{N}_+ \;\&\; s \in \mathbb{N}_+ \;\&\; (\exists l \in \mathbb{N}_+)(k = ls)\}\).
Ще докажем, че \(L\) не е регулярен като използваме следното наблюдение.

Нека \(\alpha, \beta \in \{a, b, c\}^*\). Тогава ако съществува \(\gamma \in \{a, b, c\}^*\), такава че
\(\alpha.\gamma \in L \;\&\; \beta.\gamma \notin L\), то \(\alpha\) и \(\beta\) не са в релация спрямо \(\approx_L\).

\vspace*{5mm}

\par Удобно е да конструираме по унифициран начин думи на база тяхната дължина.
За конкретният пример изображение \(word : \mathbb P \to \{a, b, c\}^*\), такова че \(word(p) = a^p.c^p\) ще ни свърши работа.
То очевидно е инективно, тоест по различен параметър, в случая просто число ни дава различна дума!
Нека \(p\) и \(q\) са две различни прости числа. Ще покажем, че \(word(p)\) и \(word(q)\) не са в релация.
Тоест \(\lnot \; c^p.a^p \approx_L c^q.a^q \). За целта трябва да посочим дума \(\gamma \in \{a, b, c\}^*\),
такава че \(c^p.a^p.\gamma \in L\), но \(c^q.a^q.\gamma \notin L\).
Очевидо \(b^p \in \{a, b, c\}^*\) и \(c^pa^pb^p \in L\), защото \(p = 1.p\) и \(p \in \mathbb{N}_+\).
Но \((c^q.a^q).b^p \notin L\), защото уравнението \(q = xp\) няма целочислено решение.

\vspace*{3mm}

\par Една от теоремите, които Евклид е доказал гласи, че множеството на простите числа не е крайно.
Но понеже \(\mathbb P \subseteq \mathbb N\), то \(\mathbb P\) е изброимо безкрайно.
Тогава изброимо безкрайно е и образа на \(\mathbb P\) под инективна функция.
Така излиза, че \(Range(word)\) е изброимо безкрайно подмножество на \(\{a, b, c\}^*\) от несравними спрямо \(\approx_L\) думи, защото \[Range(word) = word[\mathbb P] = \{a^t.b^t \mid t \in \mathbb P\}\]

Следователно \(\approx_L\) няма креан индекс. Следователно от теорамата на Майхил-Нероуд \(L\) не е регулярен.

\section{Пример 2}
Нека \(L = \{a^i.b^j.c^k \;\mid\; i \in \mathbb{N} \;\&\; j \in \mathbb{N} \;\&\; k \in \mathbb{N} \;\&\; i = j \;\lor\; j = k\}\).

\vspace*{5mm}

\par Нека \(n, k \in \mathbb{N}\) и \(n \neq k\) ще покажем, че думите \(a^{n + 1}\) и \(a^{k + 1}\) не са в релация на Майхил-Нероуд за \(L\).

\vspace*{5mm}

\par Очевидно \(b^{n + 1}.c^0 \in \{a, b, c\}^*\) и \(a^{n + 1}.(b^{n + 1}.c^0 )\in L\).
Но \(a^{k + 1}.(b^{n + 1}.c^0) \notin L\), защото \(k \neq n\) и значи \(k + 1 \neq n + 1\) и \(n + 1 \neq 0\).

\vspace*{5mm}

\par Очевидно множеството \(\{a^{s + 1} \;\mid\; s \in \mathbb{N}\}\) не е крайно.
Следователно \(\approx_L\) няма креан индекс. Следователно от теорамата на Майхил-Нероуд \(L\) не е регулярен.

\section{Пример 3}
Нека \(L = \{a^i.b^j.c^k \;\mid\; i \in \mathbb{N} \;\&\; j \in \mathbb{N} \;\&\; k \in \mathbb{N} \;\&\; i \neq j \;\lor\; j \neq k \;\lor\; i \neq k\}\).

\vspace*{5mm}

\par Нека \(n, k \in \mathbb{N}\) и \(n \neq k\) ще покажем, че думите \(a^n\) и \(a^k\) не са в релация на Майхил-Нероуд за \(L\).

\vspace*{5mm}

\par Очевидно \(b^k.c^k \in \{a, b, c\}^*\) и \(a^n.(b^k.c^k)\in L\), но \(a^k(b^k.c^k) \notin L\).

\vspace*{5mm}

\par Очевидно множеството \(\{a^s \mid s \in \mathbb{N}\}\) не е крайно.
Следователно \(\approx_L\) няма креан индекс. Следователно от теорамата на Майхил-Нероуд \(L\) не е регулярен.

\end{document}
