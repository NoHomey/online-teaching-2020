\documentclass[12pt]{article}
\usepackage{tikz}
\usepackage{tikz-qtree}
\usepackage{float}
\usepackage[utf8]{inputenc}
\usepackage[T2A]{fontenc}
\usepackage[english,bulgarian]{babel}
\usepackage{amsmath}
\usepackage{amssymb}
\usepackage{amsthm}
\usepackage{relsize}
\usepackage{euler}

\title{Безконтекстни граматики}
\author{Иво Стратев}

\begin{document}

\maketitle

\section*{Въведение}
Нека \(\Sigma\) и \(\Gamma\) са крайни азбуки, такива че \(\Sigma \cap \Gamma = \emptyset\).
Нека \(S \in \Gamma\) и нека \(R \in \mathcal{P}(\Gamma \times (\Sigma \cup \Gamma)^* )\).
Тогава \(\langle \Gamma,  \Sigma, S, R\rangle\) е безконтекстна граматика.
\(\Sigma\) ще наричаме множество на терминалите или символна азбука,
а \(\Gamma\) множество на нетерминалите или множество на променливите.
\(R\) ще наричаме множеството от правилата на граматиката.
\(S\) начален нетерминал или начална променлива.


\subsection*{Съкращения}
Ако \(\langle V, \alpha \rangle \in \Gamma \times (\Sigma \cup \Gamma)^*\), то ще пишем \(V \to \alpha\)
и ще казваме, че \(V\) може да се замени с \(\alpha\). Ако \(V \in \Gamma\), \(\alpha, \beta, \omega \in (\Sigma \cup \Gamma)^*\) и \(\langle V, \alpha \rangle, \langle V, \beta \rangle, \langle V, \omega \rangle \in R\), то ще пишем \(V \to \alpha \mid \beta \mid \omega \),
което ще ни казва, че \(V\) може да бъде заменена с коя да е от думите \(\alpha\), \(\beta\) и \(\gamma\).

\subsection*{Език на безконтекстна граматика}

\par Нека \(M\) е непразно множество и \(t \in \mathbb{N}\) тогава със \(\mathrm{seq}(t, M)\) ще означаме множеството на функциите от \(\{1, 2, \dots, t\}\) в \(M\) (крайните редиците от \(t\) на брой елемента от M).

\vspace*{5mm}
\par Нека \(G = \langle \Gamma,  \Sigma, S, R\rangle\). 
С рекурсия по естествените числа дефинираме релации \(\underset{G}{\overset{n}{\leadsto}}\)  между \(\Gamma\) и \((\Sigma \cup \Gamma)^*\).
Като идеята ни е следната искаме \\
\(V \underset{G}{\overset{n}{\leadsto}} \alpha\) ТСТК има извод на думата \(\alpha\) от променливата \(V\) с височина \(n\) по граматиката \(G\).
Дефинираме
\[\underset{G}{\overset{0}{\leadsto}} \;:=\; Id_\Gamma = \{\langle V, V \rangle \;\mid\; V \in \Gamma\}\]
Следвайки идеята ни с височина \(0\) от една промнелива можем да изведем самата променлива, защото няма как да приложим правило.
\vspace*{5mm}
\par Ако \(V \in \Gamma\), \(k \in \mathbb{N}\), \(\alpha \in \mathrm{seq}(k + 1, \Sigma^*)\), \(\beta \in \mathrm{seq}(k, (\Sigma \cup \Gamma)^*)\), \(E \in \mathrm{seq}(k, \Gamma)\), \(V \to \left(\displaystyle\prod_{i = 1}^k \alpha(i).E(i) \right).\alpha(k + 1) \in R\), \(l \in \mathrm{seq}(k, \mathbb{N})\),

\(n = \max\{ l(i) \mid i \in \{1, 2, \dots, k\} \}\) и \((\forall i \in \{1, 2, \dots, k\})(E(i) \underset{G}{\overset{l(i)}{\leadsto}} \beta(i) )\),

то \(V \underset{G}{\overset{n + 1}{\leadsto}} \left( \displaystyle\prod_{i = 1}^k \alpha(i).\beta(i) \right ).\alpha(k + 1)\).

\vspace*{5mm}
\par Тоест всеки път когато имаме променлива \(V\), редица от \(k\) променливи: \(E(1), E(2), \dots, E(k)\),
редица от \(k + 1\) думи над \(\Sigma \cup \Gamma\): \(\alpha(1), \alpha(2), \dots, \alpha(k + 1)\),
редица от \(k\) думи над \(\Sigma \cup \Gamma\): \(\beta(1), \beta(2), \dots, \beta(k)\), \\
правило \(V \to \alpha(1).E(1).\alpha(2).E(2).\alpha(3) \dots \alpha(k).E(K).\alpha(k + 1)\) в граматиката,
редица от \(k\) естесвени числа \(l(1), l(2), \dots l(k)\) и \(n\) е максималното число в крайната редица \(l\)
и имаме, че от променливата \(E(i)\) се извежда думата \(\beta(i)\) с височина \(l(i)\) за \(i\) от \(1\) до \(k\),
то от променливата \(V\) се извежда думата \(\alpha(1).\beta(1).\alpha(2).\beta(2).\alpha(3) \dots \alpha(k).\beta(K).\alpha(k + 1)\) с височина на извода \(n + 1\).
Тоест позволяваме при прилагане на едно правило да направим едновременна замяна на всяка участваща променлива в дясната част на правилото, като можем да изберем и тривиалната замяна, а именно да не правим замяна на дадена променлива. Така реално си даваме пълна свобода на действие.

\vspace*{5mm}
\par Накрая дефинираме \(\underset{G}{\overset{\star}{\leadsto}}\) като \(\displaystyle\bigcup_{s \in \mathbb{N}} \underset{G}{\overset{s}{\leadsto}}\).
Така \(V \underset{G}{\overset{\star}{\leadsto}} \alpha\) ТСТК от променливата \(V\) е изводима думата \(\alpha\) по граматиката \(G\).
Помислите защо \(\underset{G}{\overset{\star}{\leadsto}}\) реално играе ролята на \textbf{рефлексивното} и \textbf{транзитивно} затваряне на релацията определена от множеството \(R\).

\subsubsection*{Пример}
Нека \(\Gamma = \{S, D, B\}\), \(\Sigma = \{d, b\}\) и

\(R = \{ S \to BDS \mid B, D \to dDb \mid d, B \to bdB \mid b \}\).

Разглеждаме граматиката \(\langle \Gamma, \Sigma, S, R \rangle\).

Имаме \(B \overset{1}{\leadsto} b\)  и \(D \overset{1}{\leadsto} d\). Значи \(B \overset{1}{\leadsto} b\) и \(D \overset{1 + 1}{\leadsto} ddb\).

На диаграма:

\begin{figure}[H]
\centering
\begin{tikzpicture}
\Tree
[.$D$
    $d$
    [.$D$ $d$ ]
    $b$
]
\end{tikzpicture}
\caption{Дърво на извод за думата $ddb$}
\end{figure}

Вижда се, че в дясната част на правилото \(D \to dDb\) сме заменили \(D\) с \(d\).
Тоест за извода на думата \(ddb\) от промеливата \(D\) сме приложили последователно правилата \(D \to dDb\) и \(D \to d\).

\vspace*{5mm}
\par Имаме правило \(S \to B\) и \(B \overset{1}{\leadsto} b\) следователно \(S \overset{2}{\leadsto} b\). \\
Така имаме \(B \overset{1}{\leadsto} b\), \(D \overset{2}{\leadsto} ddb\), \(S \overset{2}{\leadsto} b\) и правило \(S \to BDS\),
получаваме \(S \overset{1 + max(1, 2, 2)}{\leadsto} b(ddb)b\).

\vspace*{5mm}
\par До тук имаме \(B \overset{1}{\leadsto} b\),  \(D \overset{2}{\leadsto} ddb\), \(S \overset{3}{\leadsto} bddbb\) и правило \(S \to BDS\),
получаваме \(S \overset{1 + max(1, 2, 3)}{\leadsto} b(ddb)(bddbb)\). Тоест \(S \overset{4}{\leadsto} bddbbddbb\).
Получихме, че от \(S\) с височина на извода \(4\) се извежда думата \(bddbbddbb\).

\begin{figure}[H]
\centering
\begin{tikzpicture}
\Tree
[.$S$
    [.$B$ $b$ ]
    [.$D$
        $d$
        [.$D$ $d$ ]
        $b$
    ]
    [.$S$
        [.$B$ $b$ ]
        [.$D$
            $d$
            [.$D$ $d$ ]
            $b$
        ]
        [.$S$ [.$B$ $b$ ] ]
    ]
]
\end{tikzpicture}
\caption{Дърво на извод за думата $bddbbddbb$}
\end{figure}

\subsubsection*{Дефиниция (език на променлива)}
Нека \(G = \langle \Gamma,  \Sigma, S, R\rangle\) е безконтекстна граматика.
Нека \(V \in \Gamma\). Тогава езикът на \(V\) спрямо \(G\) е \(\{\omega \in \Sigma^* \mid V \underset{G}{\overset{\star}{\leadsto}} \omega\}\).
Ще го белжим с \(\mathcal{CFL}_G(V)\). Така \(\mathcal{CFL}_G(V)\) е множеството на думите над символната азбука, които са изводими по граматиката \(G\) от променливата \(V\).

\subsubsection*{Дефиниция (език на безконтекстна граматика)}
Тогава езикът на \(G\) е точно \(\mathcal{CFL}_G(S)\) и ще го бележим с \(\mathcal{CFL}(G)\).
Тоест езикът на граматиката \(G\) е езикът на началната променлива на граматиката.

\subsubsection*{Дефиниция (крайна апроксимация на език на променлива)}
Нека \(G = \langle \Gamma,  \Sigma, S, R\rangle\) е безконтекстна граматика.
Нека \(V \in \Gamma\). Нека \(l \in \mathbb{N}\) и нека \(\mathcal{CFL}_G^l(V) := \{\omega \in \Sigma^* \mid (\exists s \in \mathbb{N})(s \leq l \;\&\; V \underset{G}{\overset{s}{\leadsto}} \; \omega)\}\). Множеството \(\mathcal{CFL}_G^l(V)\) е \(l\)-тата крайна апроксимация на езика \(\mathcal{CFL}_G(V)\).
Това са думите над символната азбука, изводими от \(V\) с височина на извода не по-голяма от \(l\).
Понеже с височина на извода \(0\) от \(V\) се извежда само \(V\) и \(V \notin \Sigma\), то \(\mathcal{CFL}_G^l(V) = \emptyset\).

\subsubsection*{Основна задача :)}
Нека \(G = \langle \Gamma,  \Sigma, S, R\rangle\) е безконтекстна граматика.
Нека \(V \in \Gamma\). Тогава
\[\mathcal{CFL}_G(V) = \displaystyle\bigcup_{n \in \mathbb{N}} \mathcal{CFL}_G^n(V)\]

\textbf{Доказателство:}
\begin{align*}
    \mathcal{CFL}_G(V) \\
    = \{\omega \in \Sigma^* \mid V \; \underset{G}{\overset{\star}{\leadsto}} \; \omega\} \\
    = \left\{\omega \in \Sigma^* \mid V \; \displaystyle\bigcup_{k \in \mathbb{N}} \underset{G}{\overset{k}{\leadsto}} \; \omega \right\} \\
    = \{\omega \in \Sigma^* \mid (\exists k \in \mathbb{N})(V \; \underset{G}{\overset{k}{\leadsto}} \; \omega)\} \\
    = \displaystyle\bigcup_{k \in \mathbb{N}} \{\omega \in \Sigma^* \mid V \; \underset{G}{\overset{k}{\leadsto}} \; \omega\} \\
    = \displaystyle\bigcup_{k \in \mathbb{N}} \{\omega \in \Sigma^* \mid (\exists l \in \mathbb{N})(l \leq k \;\&\; V \; \underset{G}{\overset{l}{\leadsto}} \; \omega)\} \\
    = \displaystyle\bigcup_{k \in \mathbb{N}} \{\omega \in \Sigma^* \mid \omega \in \mathcal{CFL}_G^k(V)\} \\
    = \displaystyle\bigcup_{k \in \mathbb{N}} \mathcal{CFL}_G^k(V) 
\end{align*}

\subsubsection*{Индуктивен принцип}
Нека \(G = \langle \Gamma,  \Sigma, S, R\rangle\) е безконтекстна граматика.
Нека \(P\) е свойство на думите над \(\Sigma\). Тоест \(P\) е предикат над \(\Sigma^*\).
Тогава е в сила следната импликация (индуктивен принцип)
\[((\forall k \in \mathbb N)(\forall \omega \in \mathcal{CFL}_G^k(V)) \; P(\omega)) \implies ((\forall \omega \in \mathcal{CFL}_G(V)) \; P(\omega))\]

\textbf{Доказателство:}
Нека е в сила предпоставката на импликацията.
Тоест нека \((\forall k \in \mathbb N)(\forall \omega \in \mathcal{CFL}_G^k(V)) \; P(\omega)\) е истина.
Ще докажем, че е истина и \((\forall \omega \in \mathcal{CFL}_G(V)) \; P(\omega)\).
Нека \(\omega \in \mathcal{CFL}_G(V)\). Тогава по основната задача \(\omega \in \displaystyle\bigcup_{n \in \mathbb{N}} \mathcal{CFL}_G^n(V)\).
Нека тогава \(n \in \mathbb N\) е такова, че \(\omega \in \mathcal{CFL}_G^n(V)\). Тогава \(P(\omega)\) от предпостаката.
Обобщаваме и получаваме \((\forall \omega \in \mathcal{CFL}_G(V)) \; P(\omega)\).

\subsubsection*{Следствие:}
Искаме да докажем, че \(\mathcal{CFL}_G(V) \subseteq L\).
Тогава вземаме свойство \\
\(P(\omega) \overset{def}{\iff} \omega \in L\).
Доказваме по индукция
\[(\forall k \in \mathbb N)(\forall \omega \in \mathcal{CFL}_G^k(V))(\omega \in L) \]
От където от индуктивният принцип получаваме \\
\((\forall \omega \in \mathcal{CFL}_G(V))(\omega \in L)\).
Тоест \(\mathcal{CFL}_G(V) \subseteq L\).

\section*{Два основни примера за безкрайни безконтекстни езици}

\subsection*{Пример 1. Всевъзжмони конкатенации (итерации) на дума / звезда на Клини на език от една дума}

Нека \(\Sigma\) е азбука. Нека \(\alpha \in \Sigma^*\). Нека \(S \notin \Sigma\).

Нека \(G = \langle \{S\}, \Sigma,  S, \{S \to \alpha.S,\; S \to \varepsilon\} \rangle\).

Твърдим, че \(\mathcal{CFL}(G) = \{\alpha\}^*\).

\vspace*{5mm}

\par С индукция ще докажем, че \((\forall k \in \mathbb N)(\mathcal{CFL}_G^k(S) \subseteq \{\alpha\}^*  )\), \\
доказвайки \((\forall k \in \mathbb N)(\forall \omega \in \mathcal{CFL}_G^k(S))(\omega \in \{\alpha\}^*  )\).

\vspace*{5mm}

\par Имаме следната рекуретна връзка
\begin{align*}
    \mathcal{CFL}_G^0(S) = \emptyset \\
    (\forall n \in \mathbb N)(\mathcal{CFL}_G^{n + 1}(S) = \{\alpha\} \cdot \mathcal{CFL}_G^n(S) \cup \{\varepsilon\})
\end{align*}

\subsubsection*{База:}
\(\mathcal{CFL}_G^0(S) = \emptyset \subseteq \{\alpha\}^*\).

\subsubsection*{И.Х.}
Нека \(k \in \mathbb N\) и нека \((\forall \omega \in \mathcal{CFL}_G^k(S))(\omega \in \{\alpha\}^*  )\). \\
Тоест нека \(\mathcal{CFL}_G^k(S) \subseteq \{\alpha\}^*\).

\subsubsection*{И.С.}
Нека \(\omega \in \mathcal{CFL}_G^{k + 1}(S)\). Тогава от връзката следва, че \\
\(\omega \in \{\alpha\} \cdot \mathcal{CFL}_G^k(S) \cup \{\varepsilon\}\).
Възможи са два случая. \\
Ако \(\omega \in \{\varepsilon\}\), то \(\omega = \alpha^0 \in \{\alpha\}^*\). \\
Ако \(\omega \in \{\alpha\} \cdot \mathcal{CFL}_G^k(S)\), то \(\omega \in \{\alpha\} \cdot \{\alpha\}^* = \{\alpha\}^+ \subseteq \{\alpha\}^*\).
Така получихме, че \(\omega \in \{\alpha\}^*\).
Обобщаваме и получаваме \((\forall \omega \in \mathcal{CFL}_G^{k + 1}(S))(\omega \in \{\alpha\}^*)\).

\subsubsection*{Заключение:}
\((\forall k \in \mathbb N)(\forall \omega \in \mathcal{CFL}_G^k(S))(\omega \in \{\alpha\}^*  )\).
Така от следствивето на индукционни принцип получаваме \(\mathcal{CFL}_G(S) \subseteq \{\alpha\}^*\).

\vspace*{5mm}

\par Сега ще докажем обратното включване. Тоест \((\forall \omega \in \{\alpha\}^*)(\omega \in \mathcal{CFL}_G(S))\).
Понеже \(\{\alpha\}^* = \{\alpha^n \mid n \in \mathbb N\}\) ще докажем 
\((\forall n \in \mathbb N)(\alpha^n \in \mathcal{CFL}_G(S))\).
Като вземем предвид, че \(\mathcal{CFL}_G(S) = \{\omega \in \Sigma^* \mid S \overset{*}{\leadsto} \omega \}\),
което е еквивалетно с \((\forall n \in \mathbb N)(\exists k \in \mathbb N)(S \overset{k}{\leadsto} \alpha^n)\).

\subsubsection*{База:}
Имаме правило \(S \to \varepsilon\) следователно \(S \overset{1}{\leadsto} \alpha^0\).
Следователно \(\alpha^0 \in \mathcal{CFL}_G(S)\).

\subsubsection*{И.Х.}
Нека \(n \in \mathbb N\) и нека \(\alpha^n \in \mathcal{CFL}_G(S)\) т.е. \(S \overset{*}{\leadsto} \alpha^n \).

\subsubsection*{И.С.}
Щом \(S \overset{*}{\leadsto} \alpha^n \), то нека \(l \in \mathbb N\) е такова, че \(S \overset{l}{\leadsto} \alpha^n\).
Имаме правило \(S \to \alpha S\). Следователно \(S \overset{1 + l}{\leadsto} \alpha.\alpha^n\).
Така \(S \overset{1 + l}{\leadsto} \alpha^{n + 1}\). Следователно \(S \overset{*}{\leadsto} \alpha^{n + 1}\).
Следователно \(\alpha^{n + 1} \in \mathcal{CFL}_G(S)\).

\subsubsection*{Заключение:}
\((\forall n \in \mathbb N)(\alpha^n \in \mathcal{CFL}_G(S))\).
Следователно \(\{\alpha\}^* \subseteq \mathcal{CFL}_G(S)\).

Така \(\mathcal{CFL}(G) = \mathcal{CFL}_G(S) = \{\alpha\}^*\).

\subsection*{Пример 2. Основен безконтекстен}

Нека \(\Sigma\) е азбука. Нека \(\alpha, \beta \in \Sigma^*\). Нека \(S \notin \Sigma\).

Нека \(G = \langle \{S\}, \Sigma,  S, \{S \to \alpha.S.\beta,\; S \to \varepsilon\} \rangle\).

Твърдим, че \(\mathcal{CFL}(G) = \{\alpha^n \beta^n \mid n \in \mathbb N\}\).

Нека \(L = \{\alpha^n \beta^n \mid n \in \mathbb N\}\).
Трябва да покажем

\(\mathcal{CFL}_G(S) \subseteq L\) и \(L \subseteq \mathcal{CFL}_G(S)\).

\vspace*{5mm}

\par С индукция ще докажем, че \((\forall k \in \mathbb N)(\mathcal{CFL}_G^k(S) \subseteq L  )\), \\
доказвайки \((\forall k \in \mathbb N)(\forall \omega \in \mathcal{CFL}_G^k(S))(\omega \in L )\).

\vspace*{5mm}

\par Имаме следната рекуретна връзка
\begin{align*}
    \mathcal{CFL}_G^0(S) = \emptyset \\
    (\forall n \in \mathbb N)(\mathcal{CFL}_G^{n + 1}(S) = (\{\alpha\} \cdot \mathcal{CFL}_G^n(S) \cdot \{\beta\}) \cup \{\varepsilon\})
\end{align*}

\subsubsection*{База:}
\(\mathcal{CFL}_G^0(S) = \emptyset \subseteq L\).

\subsubsection*{И.Х.}
Нека \(k \in \mathbb N\) и нека \((\forall \omega \in \mathcal{CFL}_G^k(S))(\omega \in L  )\). \\
Тоест нека \(\mathcal{CFL}_G^k(S) \subseteq L\).

\subsubsection*{И.С.}
Нека \(\omega \in \mathcal{CFL}_G^{k + 1}(S)\). Тогава от връзката следва, че \\
\(\omega \in (\{\alpha\} \cdot \mathcal{CFL}_G^k(S) \cdot \{\beta\}) \cup \{\varepsilon\}\).
Възможи са два случая. \\
Ако \(\omega \in \{\varepsilon\}\), то \(\omega = \alpha^0\beta^0 \in L\). \\
Ако \(\omega \in \{\alpha\} \cdot \mathcal{CFL}_G^k(S) \cdot \{\beta\}\), то \(\omega \in \{\alpha\} \cdot L \cdot \{\beta\} = \{\alpha^{n + 1} \beta^{n + 1} \mid n \in \mathbb N\} \subseteq L\).
Така получихме, че \(\omega \in L\).
Обобщаваме и получаваме

\((\forall \omega \in \mathcal{CFL}_G^{k + 1}(S))(\omega \in L)\).

\subsubsection*{Заключение:}
\((\forall k \in \mathbb N)(\forall \omega \in \mathcal{CFL}_G^k(S))(\omega \in L )\).
Така от следствивето на индукционни принцип получаваме \(\mathcal{CFL}_G(S) \subseteq L\).

\vspace*{5mm}

\par Сега ще докажем обратното включване. Тоест \((\forall \omega \in L)(\omega \in \mathcal{CFL}_G(S))\).
Понеже \(L= \{\alpha^n \beta^n \mid n \in \mathbb N\}\) ще докажем 
\((\forall n \in \mathbb N)(\alpha^n \beta^n \in \mathcal{CFL}_G(S))\).
Като вземем предвид, че \(\mathcal{CFL}_G(S) = \{\omega \in \Sigma \mid S \overset{*}{\leadsto} \omega \}\),
което е еквивалетно с \((\forall n \in \mathbb N)(\exists k \in \mathbb N)(S \overset{k}{\leadsto} \alpha^n \beta^k)\).

\subsubsection*{База:}
Имаме правило \(S \to \varepsilon\) следователно \(S \overset{1}{\leadsto} \alpha^0 \beta^0\).

Следователно \(\alpha^0 \beta^0 \in \mathcal{CFL}_G(S)\).

\subsubsection*{И.Х.}
Нека \(n \in \mathbb N\) и нека \(\alpha^n \beta^n \in \mathcal{CFL}_G(S)\) т.е. \(S \overset{*}{\leadsto} \alpha^n \beta^n \).

\subsubsection*{И.С.}
Щом \(S \overset{*}{\leadsto} \alpha^n \beta^n \), то нека \(l \in \mathbb N\) е такова, че \(S \overset{l}{\leadsto} \alpha^n \beta^n\).

Имаме правило \(S \to \alpha S \beta\). Следователно \(S \overset{1 + l}{\leadsto} \alpha(\alpha^n \beta^n) \beta\).

Така \(S \overset{1 + l}{\leadsto} \alpha^{n + 1} \beta^{n + 1}\). Следователно \(S \overset{*}{\leadsto} \alpha^{n + 1} \beta^{n + 1}\).

Следователно \(\alpha^{n + 1} \beta^{n + 1} \in \mathcal{CFL}_G(S)\).

\subsubsection*{Заключение:}
\((\forall n \in \mathbb N)(\alpha^n \beta^n \in \mathcal{CFL}_G(S))\).
Следователно \(L \subseteq \mathcal{CFL}_G(S)\).

Така \(\mathcal{CFL}(G) = \mathcal{CFL}_G(S) = L = \{\alpha^n \beta^n \mid n \in \mathbb N\}\).

\section*{Операции с граматики}

\subsection*{Сума на граматики}
Нека \(G_1 = \langle \Gamma_1, \Sigma_1, S_1, R_1 \rangle\) и \(G_2 = \langle \Gamma_2, \Sigma_2, S_2, R_2 \rangle\) са КСГ и \(\Gamma_1 \cap \Gamma_2 = \emptyset\).

Нека \(S = \langle \Gamma_1, \Gamma_2 \rangle\) очевидно \(S \notin \Gamma_1 \cup \Gamma_2\) и

\(G_1 \oplus G_2 := \langle \Gamma_1 \cup \Gamma_2 \cup \{S\},\; \Sigma_1 \cup \Sigma_2,\; S,\; R_1 \cup R_2 \cup \{S \to S_1, S \to S_2\} \rangle\).

Тогава \(\mathcal{CFL}(G_1 \oplus G_2) = \mathcal{CFL}(G_1) \cup \mathcal{CFL}(G_2)\).

\(G_1 \oplus G_2\) е сумата на граматиките \(G_1\) и \(G_2\).

Очевидно \(G_2 \oplus G_1 \neq G_1 \oplus G_2\), защото 
\(\langle \Gamma_1, \Gamma_2 \rangle \neq \langle \Gamma_2, \Gamma_1 \rangle\),

но \(\mathcal{CFL}(G_2 \oplus G_1) = \mathcal{CFL}(G_1 \oplus G_2)\)!

\subsection*{Произведение на граматики}
Нека \(G_1 = \langle \Gamma_1, \Sigma_1, S_1, R_1 \rangle\) и \(G_2 = \langle \Gamma_2, \Sigma_2, S_2, R_2 \rangle\) са КСГ и \(\Gamma_1 \cap \Gamma_2 = \emptyset\).


Нека \(S = \langle \Gamma_1, \Gamma_2 \rangle\) очевидно \(S \notin \Gamma_1 \cup \Gamma_2\) и


\(G_1 \odot G_2 := \langle \Gamma_1 \cup \Gamma_2 \cup \{S\},\; \Sigma_1 \cup \Sigma_2,\; S,\; R_1 \cup R_2 \cup \{S \to S_1S_2\} \rangle\).

Тогава \(\mathcal{CFL}(G_1 \odot G_2) = \mathcal{CFL}(G_1) \cdot \mathcal{CFL}(G_2)\).

\(G_1 \odot G_2\) е произведението на граматиките \(G_1\) и \(G_2\).

\subsection*{Звезда (итерация) на безконтекстна граматика}

Нега \(G =  \langle \Gamma, \Sigma, S, R \rangle\).

Нека \(P = \Gamma\), очевидно \(P \notin \Gamma\).

Нека \(G^\circledast := \langle \Gamma \cup \{P\},\; \Sigma,\; P,\; R \cup \{P \to S.P,\; P \to \varepsilon\} \rangle\).

Тогава \(\mathcal{CFL}(G^\circledast) = \mathcal{CFL}(G)^*\).

\section*{Пример 1}
Нека \(L = \{a^n.b^k.c^s \mid n \in \mathbb{N} \;\&\; k \in \mathbb{N} \;\&\; s \in \mathbb{N} \;\&\; n + k \leq s\}\).
Искаме да покажем, че \(L\) е безконтекстен език. За да си цел трябва да построим безконтекстна граматика, която го генерира.

Преди да строим граматика ще изразим \(L\) в удобен за генериране вид.
Възползваме се от практическото правило, че лесно се правят правила за генерира от вън на вътре!

\begin{align*}
    L = \{a^n.b^k.c^{n + k + t} \mid n \in \mathbb{N} \;\&\; k \in \mathbb{N} \;\&\; t \in \mathbb{N}\} \\
    L = \{a^n.b^k.c^k.c^n.c^t \mid n \in \mathbb{N} \;\&\; k \in \mathbb{N} \;\&\; t \in \mathbb{N}\} \\
    L = \{a^n.b^k.c^k.c^n.\omega \mid n \in \mathbb{N} \;\&\; k \in \mathbb{N} \;\&\; \omega \in \{c\}^*\} \\
    L = \{a^n.(b^k.c^k).c^n \mid n \in \mathbb{N} \;\&\; k \in \mathbb{N} \} \cdot \{c\}^*
\end{align*}

Нека \(T = \{a^n.b^k.c^k.c^n \mid n \in \mathbb{N} \;\&\; k \in \mathbb{N} \}\). Тогава \(L = T \cdot \{c\}^*\).

Нека \(G_1 = \langle \{A, B\}, \{a, b\},  A, \{A \to aAc,\; A \to B,\; B \to bBc,\; B \to \varepsilon\} \rangle\).

Тогава \(\mathcal{CFL}_{G_1}(B) = \mathcal{CFL}_{\langle \{B\}, \{b, c\}, B, \{B \to bBc,\; B \to \varepsilon\} \rangle}(B)
= \{b^k.c^k \mid k \in \mathbb{N}\}\).

Така получаваме

\((\forall n \in \mathbb{N})(\mathcal{CFL}_{G_1}^n(B) = \{b^s.c^s \mid s \in \mathbb{N} \;\&\; s < n\} \subseteq \{b^k.c^k \mid k \in \mathbb{N}\} \subseteq T)\).

\vspace*{5mm}

\par В сила е следната рекуретна връзка между крайните апроксимации на \(\mathcal{CFL}_{G_1}(A)\).

\begin{align*}
    \mathcal{CFL}_{G_1}^0(A) = \emptyset\ \\
    (\forall n \in \mathbb N)(\mathcal{CFL}_{G_1}^{n + 1}(A) = (\{a\} \cdot \mathcal{CFL}_{G_1}^n(A) \cdot \{c\}) \cup \mathcal{CFL}_{G_1}^n(B))
\end{align*}

С индукция ще докажем, че

\((\forall h \in \mathbb{N})(\mathcal{CFL}_{G_1}^h(A) \subseteq T = \{a^n.b^k.c^k.c^n \mid n \in \mathbb{N} \;\&\; k \in \mathbb{N}\})\).

\subsection*{База:}
Имаме \(\mathcal{CFL}_{G_1}^0(A) = \emptyset \subseteq T\).

\subsection*{И.Х.}
Нека \(h \in \mathbb{N}\) и \(\mathcal{CFL}_{G_1}^h(A) \subseteq T = \{a^n.b^k.c^k.c^n \mid n \in \mathbb{N} \;\&\; k \in \mathbb{N}\}\).

\subsection*{И.С.}
Нека \(\omega \in \mathcal{CFL}_{G_1}^{h + 1}(A)\). Възможни са два случая.

\subsubsection*{Случай 1. \(\omega \in \{a\} \cdot \mathcal{CFL}_{G_1}^h(A) \cdot \{c\}\)}
Тогава нека \(\beta \in \mathcal{CFL}_{G_1}^h(A)\) и \(\omega = a.\beta.c\).
От хипотезата получаваме \((\exists n \in \mathbb{N})(\exists k \in \mathbb{N})(\beta = a^n.b^k.c^k.c^n)\).
Нека тогава \(n, k \in \mathbb{N}\) и са такива, че \(\beta = a^n.b^k.c^k.c^n\).
Така \(\omega = a.\beta.c = a^{n + 1}.b^k.c^k.c^{n + 1} \in T\).


\subsubsection*{Случай 2. \(\omega \in \mathcal{CFL}_{G_1}^h(B)\)}
Тогава \(\omega \in \{b^k.c^k \mid k \in \mathbb{N}\} \subseteq T\).

\vspace*{5mm}

\par Следователно \(\omega \in T\) и значи \(\mathcal{CFL}_{G_1}^{h + 1}(A) \subseteq T\).

\subsection*{Заключение:}
\((\forall k \in \mathbb N)(\forall \omega \in \mathcal{CFL}_{G_1}^k(A))(\omega \in T )\).
Така от следствивето на индукционни принцип получаваме \(\mathcal{CFL}_{G_1}(A) \subseteq T\).

\vspace*{5mm}

\par Сега ще покажем обратното включване чрез индукция по дължината на думата (брой букви \(a\)).
Твърдението, което ще докажем е 

\((\forall n \in \mathbb{N})(\forall k \in \mathbb{N})(a^n.b^k.c^k.c^n \in \mathcal{CFL}(G_1))\).

\subsection*{База:}
Нека \(k \in \mathbb{N}\). Имаме \(B \overset{k + 1}{\leadsto} b^k.c^k\).
Имаме правило \(A \to B\). Следователно \(A \overset{1 + k + 1}{\leadsto} b^k.c^k\).
Следователно \(a^0.b^k.c^k.c^0 \in \mathcal{CFL}_{G_1}(A)\).

Следователно \((\forall k \in \mathbb{N})(a^0.b^k.c^k.c^0 \in \mathcal{CFL}_{G_1}(A))\).

\subsection*{И.Х.}
Нека \(n \in \mathbb{N}\) и нека \((\forall k \in \mathbb{N})(a^n.b^k.c^k.a^n \in \mathcal{CFL}(G_1))\).

\subsection*{И.С.}
Стъпка. Нека \(k \in \mathbb{N}\). От хипотезата \(a^n.b^k.c^k.c^n \in \mathcal{CFL}(G_1)\).

Нека тогава \(l \in \mathbb{N}\) е такова, че \(A \overset{l}{\leadsto} a^n.b^k.c^k.c^n\).

Имаме правило \(A \to aAc\), следователно \(A \overset{1 + l}{\leadsto} a^{n + 1}.b^k.c^k.c^{n + 1}\).

Следователно \((\forall k \in \mathbb{N})(a^{n + 1}.b^k.c^k.c^{n + 1} \in \mathcal{CFL}(G_1))\).

\subsection*{Заключение:}
\((\forall n \in \mathbb{N})(\forall k \in \mathbb{N})(a^n.b^k.c^k.c^n \in \mathcal{CFL}(G_1))\).

Следователно \(T \subseteq \mathcal{CFL}(G_1)\). Следователно \(\mathcal{CFL}(G_1) = T\).

\subsection*{Конструиране на граматика}
Нека \(G_2 = \langle \{C\}, \{c\},  C, \{C \to cC,\; C \to \varepsilon\} \rangle\).
Тогава \(\mathcal{CFL}(G_2) = \{c\}^*\).

Нека \(G = G_1 \odot G_2\).
Тогава

\(\mathcal{CFL}(G) = \mathcal{CFL}(G_1 \odot G_2) = \mathcal{CFL}(G_1) \cdot \mathcal{CFL}(G_2) = T \cdot \{c\}^* = L\).

Следователно \(L\) е безконтекстен.

\section*{Пример 2}
Нека \(L = \{a^n.b^k.c^s \mid n \in \mathbb{N} \;\&\; k \in \mathbb{N} \;\&\; s \in \mathbb{N} \;\&\; n + k \geq s\}\).
Ще докажем, че \(L\) е безконтекстен.
Първо ще изразим \(L\) в удобен за генериране вид като отново следваме правилото: Лесно се генерира от вън на вътре.

\begin{align*}
    L = \{a^n.b^k.c^s \mid n \in \mathbb{N} \;\&\; k \in \mathbb{N} \;\&\; s \in \mathbb{N} \;\&\; (\exists t \in \mathbb{N})(n + k = s + t )\} \\
    L = \{a^n.b^k.c^{s_a + s_b} \mid n, k, s_a, s_b \in \mathbb{N} \;\&\; (\exists t \in \mathbb{N})(n + k = s_a + s_b + t )\} \\
    L = \{a^n.b^k.c^{s_a + s_b} \mid n, k, s_a, s_b \in \mathbb{N} \;\&\; (\exists t_a, t_b \in \mathbb{N})(n + k = s_a + s_b + t_a + t_b )\} \\
    L = \{a^{s_a + t_a}.b^{s_b + t_b}.c^{s_a + s_b} \mid t_a, t_b, s_a, s_b \in \mathbb{N}\} \\
    L = \{a^{s_a}.(a^{t_a}.b^{t_b}).(b^{s_b}.c^{s_b}).c^{s_a} \mid t_a, t_b, s_a, s_b \in \mathbb N\} \\
     L = \{a^t.(a^n b^k).(b^l.c^l).c^t \mid n, k, t, l \in \mathbb N\}
\end{align*}

Нека \(M = \{a\}^* \cdot \{b\}^* \cdot \{b^k.c^k \mid k \in \mathbb{N}\}\).

Тогава \(L = \{a^n.\omega.c^n \mid n \in \mathbb{N} \;\&\; \omega \in M\}\).

Нека \(G_1 = \langle \{A\}, \{a\},  A, \{A \to aA,\; A \to \varepsilon\} \rangle\).

Нека \(G_2 = \langle \{B\}, \{b\},  B, \{B \to bB,\; B \to \varepsilon\} \rangle\).

Нека \(G_3 = \langle \{F\}, \{b, c\},  F, \{F \to bFc,\; F \to \varepsilon\} \rangle\).

Нека \(G_4 = (G_1 \odot G_2) \odot G_3\). Тогава 

\(\mathcal{CFL}(G_4) = \mathcal{CFL}((G_1 \odot G_2) \odot G_3) = (\mathcal{CFL}(G_1) \cdot \mathcal{CFL}(G_2)) \cdot \mathcal{CFL}(G_3) = \{a\}^* \cdot \{b\}^* \cdot \{b^k.c^k \mid k \in \mathbb{N}\} = M\).

Ще хакнем граматиката за \(G_4\), която е за \(M\) за да направим граматика за \(L\).
Нека \(\langle \Gamma,\; \{a, b, c\},\; I,\; R \rangle = G_4\) (декомпозираме \(G_4\)).

Нека \(G = \langle \Gamma \cup \{S\},\; \{a, b, c\},\; S,\; R \cup \{S \to aSc,\; S \to I\} \rangle\).

В сила е \(\mathcal{CFL}_G(I) = \mathcal{CFL}_{G_4}(I) = \mathcal{CFL}(G_4) = \\ \{a\}^* \cdot \{b\}^* \cdot \{b^k.c^k \mid k \in \mathbb{N}\} = M \subseteq L\)

В сила е и следната рекуретна връзка

\begin{align*}
    \mathcal{CFL}_{G}^0(S) = \emptyset \\
    (\forall n \in \mathbb N)(\mathcal{CFL}_{G}^{n + 1}(S) = \{a\} \cdot \mathcal{CFL}_{G}^n(S) \cdot \{c\} \cup \mathcal{CFL}_{G}^n(I))
\end{align*}

С индукция ще докажем, че \((\forall h \in \mathbb{N})(\mathcal{CFL}_{G}^h(S) \subseteq L)\).

\subsection*{База:}
Имаме \(\mathcal{CFL}_G^0(S) = \emptyset \subseteq L\).

\subsection*{И.Х.}
Нека \(h \in \mathbb{N}\) и \(\mathcal{CFL}_{G}^h(S) \subseteq L\).

\subsection*{И.С.}
Нека \(\omega \in \mathcal{CFL}_{G}^{h + 1}(S)\). Възможни са два случая.

\subsubsection*{Случай 1. \(\omega \in \{a\} \cdot \mathcal{CFL}_{G}^h(S) \cdot \{c\} \subseteq \{a\} \cdot L \cdot \{c\}\)}

Нo \(L = \{a^n.b^k.c^s \mid n \in \mathbb{N} \;\&\; k \in \mathbb{N} \;\&\; s \in \mathbb{N} \;\&\; n + k \geq s\}\)
и \(\{a\} \cdot L \cdot \{c\} = \{a^{n + 1}.b^k.c^{s + 1} \mid n \in \mathbb{N} \;\&\; k \in \mathbb{N} \;\&\; s \in \mathbb{N} \;\&\; n + 1 + k \geq s + 1\} \subset L\).

\subsubsection*{Случай 2. \(\omega \in \mathcal{CFL}_{G}^h(I)\)}
Тогава \(\omega \in M \subseteq L\).

\vspace*{5mm}

\par Следователно \(\omega \in L\) и значи \(\mathcal{CFL}_{G}^{h + 1}(S) \subseteq L\).

\subsection*{Заключение:}
\((\forall k \in \mathbb N)(\forall \omega \in \mathcal{CFL}_G^k(S))(\omega \in L )\).
Така от следствивето на индукционни принцип получаваме \(\mathcal{CFL}_G(S) \subseteq L\).

\vspace*{5mm}

\par Имаме \(L = \{a^n.\omega.c^n \mid n \in \mathbb{N} \;\&\; \omega \in M\}\).
Сега ще покажем 

\(L \subseteq \mathcal{CFL}_G(S))\) като по индукция докажем


\((\forall n \in \mathbb{N})(\forall \omega \in M)(a^n.\omega.c^n \in \mathcal{CFL}_G(S))\).

\subsection*{База.}
Нека \(\omega \in M\). Тогава \(\omega \in \mathcal{CFL}(G_4)\).
Нека тогава \(k \in \mathbb{N}\) е такова, че \(I \;\underset{G_4}{\overset{k}{\leadsto}}\; \omega\). Тогава \(I \;\underset{G}{\overset{k}{\leadsto}}\; \omega\).
Имаме правило \(S \to I\). Следователно \(S \;\underset{G}{\overset{1 + k}{\leadsto}}\; \omega\).

\vspace*{4mm}

\par Следователно \(a^0.\omega.c^0 \in \mathcal{CFL}(G)\).

Следователно \((\forall \omega \in M)(a^0.\omega.c^0 \in \mathcal{CFL}(G))\).

\subsection*{И.Х.}
Нека \(n \in \mathbb{N}\) и нека \((\forall \omega \in M)(a^n.\omega.c^n \in \mathcal{CFL}(G))\).

\subsection*{И.С.}
Нека \(\omega \in M\). От хипотезата \(a^n.\omega.c^n \in \mathcal{CFL}(G)\).

Нека тогава \(j \in \mathbb{N}\) е такова, че \(S \;\underset{G}{\overset{j}{\leadsto}}\; a^n.\omega.c^n\).

Имаме правило \(S \to aSc\), следователно \(S \;\underset{G}{\overset{1 + j}{\leadsto}}\; a^{n + 1}.\omega.c^{n + 1}\).

Следователно \((\forall \omega \in M)(a^{n + 1}.\omega.c^{n + 1} \in \mathcal{CFL}(G))\).

\subsection*{Заключение.}
\((\forall n \in \mathbb{N})(\forall \omega \in M)(a^n.\omega.c^n \in \mathcal{CFL}_G(S))\).

Следователно \(L \subseteq \mathcal{CFL}(G)\). Значи \(\mathcal{CFL}(G) = L\).

Следователно \(L\) е безконтекстен.

\section*{Пример 3}
Нека \(L = \{\alpha\#\beta \mid \alpha, \beta \in \{a, b\}^+ \;\&\; (\exists i \in \mathbb N)(\alpha_i \neq \beta_i)\}\).

Ще покажем, че \(L\) е безконтекстен. Започваме с анализ на езика.

\vspace*{5mm}

\par Нека \(flip : \{a, b\} \to \{a, b\} \) е такава, че \(flip(a) = b\) и \(flip(b) = a\).

\vspace*{5mm}

\par Ако \(\alpha, \beta \in \{a, b\}^+\) и \((\exists i \in \mathbb N)(\alpha_i \neq \beta_i)\), то съществуват \(\rho \in \{a, b\}^*\) и \(u \in \{a, b\}\) такива, че \(\rho.u\) е префикс на \(\alpha\) и \(\rho . flip(u)\) е префикс на \(\beta\), тоест има позиция, на която буквите са различни. Но също така е вярно и, че aко \(\alpha, \beta \in \{a, b\}^+\) и \((\exists i \in \mathbb N)(\alpha_i \neq \beta_i)\), то съществуват \(\gamma, \omega \in \{a, b\}^*\) и \(u \in \{a, b\}\) такива,
че \(|\gamma| = |\omega|\) и  \(\gamma.u\) е префикс на \(\alpha\) и \(\omega . flip(u)\) е префикс на \(\beta\).
Тоест ако \(\alpha\) и \(\beta\) са различни думи, то трябва да се различават поне на някоя позиция. Ще го докажем формално.
Тоест ще докажем, че
\(L = \{\alpha\#\beta \mid \alpha, \beta \in \{a, b\}^+ \;\&\; \{a, b\}^+ \;\&\; (\exists i \in \mathbb N)(\alpha_i \neq \beta_i)\} = \\ \{(\alpha.u.\gamma\#\beta).flip(u).\omega \mid \alpha, \beta, \gamma, \omega \in \{a, b\}^* \;\&\; u \in \{a, b\} \;\&\; |\alpha| = |\beta|\}\).

\subsection*{Първо включването \(\subseteq\)}
Нека \(\alpha, \beta \in \{a, b\}^+\) и нека \((\exists i \in \mathbb N)(\alpha_i \neq \beta_i)\).
Нека тогава

\(i = min \{k \in \{1, 2, \dots min(|\alpha|, |\beta|)\} \mid \alpha_i \neq \beta_i \}\).

Тогава \((\forall j \in \{1, 2, \dots, i - 1\})(\alpha_j = \beta_j)\).
Нека тогава \(\gamma\) е префикса на \(\alpha\) с дължина \(i - 1\).
Нека означим с \(x\) буквата \(\alpha_i\). Тогава нека \(\rho\) и \(\omega\)
са суфиксите на \(\alpha\) и \(\beta\) с дължини \(|\alpha| - i\) и \(|\beta| - i\) съответно.
Тогава понеже \(x = \alpha_i \neq \beta_i\) и \(\beta_i \in \{a, b\}\), то \(\beta_i = flip(x)\).
Така \(\alpha\#\beta = \gamma.x.\rho \# \gamma.flip(x).\omega\). Понеже \(|\gamma| = |\gamma|\), то
\(\alpha\#\beta\) е във втория език.

\subsection*{Включването \(\supseteq\)}
Нека \(\alpha, \beta, \gamma, \omega \in \{a, b\}^*\), нека \(u \in \{a, b\}\) и нека \(|\alpha| = |\beta|\),

тогава \(|\alpha.u.\gamma| \geq |u| = 1\) и \(|\beta.flip(u).\omega| \geq 1\).

Следователно \(\alpha.u.\gamma, \beta.flip(u).\omega \in \{a, b\}^+\). Нека \(n = |\alpha| + 1\).

Но тогава \((\alpha.u.\gamma)_n = u \neq flip(u) = (\beta.flip(u).\omega)_n\) и \(n \in \mathbb N\).

Следователно \(\alpha.u.\gamma\#\beta.flip(u).\omega \in L\).

Схематично са възможни две ситуации за думите от \(L\) представен по втория начин. 

\begin{figure}[H]
\centering
\begin{tikzpicture}
\Tree
[.$S$
    [.$A$
        $\alpha$
        [.$A$
            $a$
            [.$F$ $\gamma$ ]
            $\#$
        ]
        $\beta$
    ]
    $flip(a)$
    [.$F$ $\omega$ ]
]
\end{tikzpicture}
\caption{Случай $u = a$}
\end{figure}

\begin{figure}[H]
\centering
\begin{tikzpicture}
\Tree
[.$S$
    [.$B$
        $\alpha$
        [.$B$
            $b$
            [.$F$ $\gamma$ ]
            $\#$
        ]
        $\beta$
    ]
    $flip(b)$
    [.$F$ $\omega$ ]
]
\end{tikzpicture}
\caption{Случай $u = b$}
\end{figure}

Променливата \(S\) ще е начална, \(A\) и \(B\) ще служат за генериране на лявата част на думата, тази в която има някаква връзка и да помним, която буква сме избрали за позицията, в която се различават думите разделени от \(\#\). Променливата \(F\) ще служи за генериране на дума от \(\{a, b\}^*\). Нека \(G = \langle \{S, A, B, F, X\},\; \{a, b\},\; S,\; R \rangle\), където \(R\) е множеството от правила

\begin{align*}
    S \to AbF \mid BaF \\
    A \to XAX \mid aF\# \\
    B \to XBX \mid bF\# \\
    X \to a \mid b \\
    F \to XF \mid \varepsilon
\end{align*}

Ясно е, че \(\mathcal{CFL}_G(X) = \{a, b\}\) и \(\mathcal{CFL}_G(F) = \{a, b\}^*\)
също така

\(\mathcal{CFL}_G^0(X) = \emptyset\) и \((\forall n \in \mathbb{N})(\mathcal{CFL}_G^{n + 1}(X) = \{a, b\})\).

Нека \(var : \{a, b\} \to \{A, B\} \) е такава, че \(var(a) = A\) и \(var(b) = B\).

Нека \(u \in \{a, b\}\) и нека \(V = var(u)\). Ще докажем, че

\(\mathcal{CFL}_G(V) = \displaystyle\bigcup_{k \in \mathbb{N}} \{a, b\}^k \cdot \{u\} \cdot \{a, b\}^* \cdot \{\#\} \cdot \{a, b\}^k\).

Нека \(L_V^k = \{a, b\}^k \cdot \{u\} \cdot \{a, b\}^* \cdot \{\#\} \cdot \{a, b\}^k\) и \(L_V = \bigcup \{L_V^k  \mid k \in \mathbb{N}\} \).

По индукция доказваме, че \((\forall n \in \mathbb{N})(\mathcal{CFL}_G^n(V) \subseteq L_V)\).

Като ползваме, че \((\forall k \in \mathbb{N})(L_V^k \subseteq L_V)\).

В сила е следната рекуретна връзка
\begin{align*}
    \mathcal{CFL}_G^0(V) = \emptyset \\
    (\forall n \in \mathbb N)(\mathcal{CFL}_G^{n + 1}(V) = (\mathcal{CFL}_G^n(X) \cdot \mathcal{CFL}_G^n(V) \cdot \mathcal{CFL}_G^n(X)) \cup (\{u\} \cdot \mathcal{CFL}_G(F) \cdot \{\#\}))
\end{align*}

След заместване получаваме

\begin{align*}
    \mathcal{CFL}_G^0(V) = \emptyset \\
    \mathcal{CFL}_G^1(V) = \emptyset \\
    (\forall n \in \mathbb N)(\mathcal{CFL}_G^{n + 2}(V) = (\{a, b\} \cdot \mathcal{CFL}_G^{n + 1}(V) \cdot \{a, b\}) \cup (\{u\} \cdot \mathcal{CFL}_G^{n + 1}(F) \cdot \{\#\}))
\end{align*}

Правим индукцията.

\subsection*{База.}
Нека \(n \in \{0, 1\}\). Тогава \(\mathcal{CFL}_G^n(V) = \emptyset \subseteq L_V\).

\subsection*{И.Х.}
Нека \(n \in \mathbb{N}\) и \(n \geq 1\). Нека \(\mathcal{CFL}_G^n(V) \subseteq L_V\).

\subsection*{И.С.}
\(\mathcal{CFL}_G^{n + 1}(V) = \{a, b\} \cdot \mathcal{CFL}_G^n(V) \cdot \{a, b\} \cup \{u\} \cdot \mathcal{CFL}_G^n(F) \cdot \{\#\} \subseteq\)

\(\{a, b\} \cdot \left(\displaystyle\bigcup_{k \in \mathbb{N}} \{a, b\}^k \cdot \{u\} \cdot \{a, b\}^* \cdot \{\#\} \cdot \{a, b\}^k\right) \cdot \{a, b\} \cup \{u\} \cdot \{a, b\}^* \cdot \{\#\} \subseteq\)

\(\left(\displaystyle\bigcup_{k \in \mathbb{N}} \{a, b\}^{k + 1} \cdot \{u\} \cdot \{a, b\}^* \cdot \{\#\} \cdot \{a, b\}^{k + 1}\right) \cup \{\varepsilon\} \cdot \{u\} \cdot \{a, b\}^* \cdot \{\#\} \cdot \{\varepsilon\} \subseteq\)
\(L_V \cup L_V^0 \subseteq L_V \cup L_V = L_V\). Следователно \(\mathcal{CFL}_G^{n + 1}(V) \subseteq L_V\).

\subsection*{Заключение.}
\((\forall n \in \mathbb N)(\forall \omega \in \mathcal{CFL}_G^n(V))(\omega \in L_V )\).
Така от следствивето на индукционни принцип получаваме \(\mathcal{CFL}_G(V) \subseteq L_V\).

\vspace*{5mm}

\par За да докажем \(L_V \subseteq \mathcal{CFL}_G(V)\) по индукция ще докажем, че

\((\forall k \in \mathbb{N})(L_V^k \subseteq \mathcal{CFL}_G(V))\).

\subsection*{База.}
Тогава  \(L_V^0 = \{a, b\}^0 \cdot \{u\} \cdot \{a, b\}^* \cdot \{\#\} \cdot \{a, b\}^0 = \{u\} \cdot \{a, b\}^* \cdot \{\#\}\). 

Нека тогава \(\omega \in \{a, b\}^*\). Искаме да покажем, че \(V \overset{\star}{\leadsto} u.\omega.\#\).

Но \(\omega \in \mathcal{CFL}_G(F)\). Нека тогава \(t \in \mathbb{N}\) е такова, че \(F \overset{t}{\leadsto} \omega\).

Имаме правило \(V \to uF\#\). Следователно \(V \overset{1 + k}{\leadsto} u\omega\#\).

Следователно \(u\omega\# \in \mathcal{CFL}_G(V)\). Следователно \(L_V^0 \subseteq \mathcal{CFL}_G(V)\).

\subsection*{И.Х.}
Нека \(k \in \mathbb{N}\). Нека \(L_V^k \subseteq \mathcal{CFL}_G(V)\).

\subsection*{И.С.}
\(L_V^{k + 1} = \{a, b\}^{k + 1} \cdot \{u\} \cdot \{a, b\}^* \cdot \{\#\} \cdot \{a, b\}^{k + 1} = \{a, b\} \cdot L_V^k \cdot \{a, b\}\).

Нека тогава \(\omega \in L_V^k\) и нека \(x, y \in \{a, b\}\). Тогава \(x.\omega.y \in L_V^{k + 1}\).

Тогава от И.Х. \(\omega \in \mathcal{CFL}_G(V)\). Нека тогава \(m  \in \mathbb{N}\) е такова, че \(V \overset{m}{\leadsto} \omega\).

Но \(X \overset{1}{\leadsto} x, y\) и имаме правило \(V \to XVX\).

Следователно \(V \;\overset{1 + max(1, m, 1)}{\leadsto} x\omega y\). Следователно \(x \omega y \in \mathcal{CFL}_G(V)\).

\subsection*{Заключение.}
\((\forall k \in \mathbb{N})(L_V^k \subseteq \mathcal{CFL}_G(V))\).

Следователно \(L_V = \displaystyle\bigcup_{s \in \mathbb{N}} L_V^s \subseteq \mathcal{CFL}_G(V)\).

Следователно \(\mathcal{CFL}_G(V) = L_V\).

Следователно \(\mathcal{CFL}_G(A) = \displaystyle\bigcup_{k \in \mathbb{N}} \{a, b\}^k \cdot \{a\} \cdot \{a, b\}^* \cdot \{\#\} \cdot \{a, b\}^k\)

и \(\mathcal{CFL}_G(B) = \displaystyle\bigcup_{k \in \mathbb{N}} \{a, b\}^k \cdot \{b\} \cdot \{a, b\}^* \cdot \{\#\} \cdot \{a, b\}^k\).

\vspace*{5mm}

\par В сила е рекурентната връзка.

\begin{align*}
    \mathcal{CFL}_G^0(S) = \emptyset \\
    (\forall n \in \mathbb N)(\mathcal{CFL}_G^{n + 1}(S) = (\mathcal{CFL}_G^n(A) \cdot \{b\} \cdot \mathcal{CFL}_G^n(F)) \cup (\mathcal{CFL}_G^n(B) \cdot \{a\} \cdot \mathcal{CFL}_G^n(F)))
\end{align*}

Диреткно ще покажем, че \(\mathcal{CFL}_G(S) \subseteq \\
\{(\alpha.u.\gamma\#\beta).flip(u).\omega \mid \alpha, \beta, \gamma, \omega \in \{a, b\}^* \;\&\; u \in \{a, b\} \;\&\; |\alpha| = |\beta|\}= L\).

\vspace*{5mm}

\par Нека \(\omega \in \mathcal{CFL}_G(S)\).
Тогава от основната задача следва, че 

\((\exists n \in \mathbb N)(\omega \in \mathcal{CFL}^n_G(S))\).
Нека тогава \(n \in \mathbb N\) е такова, че

\(\omega \in \mathcal{CFL}^n_G(S)\).
Понеже \(\mathcal{CFL}_G^0(S) = \emptyset\), то \(n > 0\).

Нека тогава \(k = n - 1\). Тогава \(k \in \mathbb N\) и \(n = k + 1\).

Тогава \(u \in \{a, b\}\), \(V = var(u)\) и \(\omega \in \mathcal{CFL}_G^k(V) \cdot \{flip(u)\} \cdot \mathcal{CFL}_G^k(F)\).

Тогава \(\omega \in \left( \displaystyle\bigcup_{k \in \mathbb{N}} \{a, b\}^k \cdot \{u\} \cdot \{a, b\}^* \cdot \{\#\} \cdot \{a, b\}^k \right) \cdot \{flip(u)\} \cdot \{a, b\}^* \\
= \{(\alpha.u.\gamma\#\beta).flip(u).\sigma \mid \alpha, \beta, \gamma, \sigma \in \{a, b\}^* \;\&\; |\alpha| = |\beta|\} \subseteq L\).

След обобщение получаваме \((\forall \omega \in \mathcal{CFL}_G(S))(\omega \in L)\).

Следователно \(\mathcal{CFL}_G(S) \subseteq L\).

\vspace*{5mm}

\par Сега с индукция ще докажем, че \(L \subseteq \mathcal{CFL}_G(S)\). Доказвайки

\((\forall n \in \mathbb{N})(\forall u \in \{a, b\})(\forall \alpha, \beta, \gamma \in \{a, b\}^*)(|\alpha| = n \implies \\ \alpha.u.\beta\#\alpha.flip(u).\gamma \in \mathcal{CFL}_G(S))\).
Тоест ще използваме първото наблюдение за думите от \(L\) в индукцията понеже там по-лесно се параметризира дължината на думата. 

\subsection*{База.}
Трябва да покажем, че

\((\forall u \in \{a, b\})(\forall \beta, \gamma \in \{a, b\}^*)(u\beta\#flip(u)\gamma \in \mathcal{CFL}_G(S))\).

Нека \(u \in \{a, b\}\). Нека \(\beta, \gamma \in \{a, b\}^* = \mathcal{CFL}_G(F)\).

Нека \(m, t \in \mathbb{N}\) са такива, че \(F \overset{s}{\leadsto} \beta\) и \(F \overset{t}{\leadsto} \gamma\).

Нека \(V = var(u)\). Тогава имаме правило \(V \to uF\#\).

Следователно \(V \overset{1 + s}{\leadsto} u\beta\#\).

Имаме правило \(S \to V.flip(u).F\).

Следователно \(S \overset{1 + max(1 + s, t)}{\leadsto} (u\beta\#).flip(u).\gamma\).

Имаме правило \(S \to V.flip(u).F\).

Следователно \(u\beta\#.flip(u).\gamma \in \mathcal{CFL}_G(S)\).

\subsection*{И.Х.}
Нека \(n \in \mathbb{N}\) и нека

\[(\forall u \in \{a, b\})(\forall \alpha, \beta, \gamma \in \{a, b\}^*)(|\alpha| = n \implies \alpha.u.\beta\#\alpha.flip(u).\gamma \in \mathcal{CFL}_G(S))\]

\subsection*{И.Х.}
Нека \(u \in \{a, b\}\). Нека \(\alpha, \beta, \gamma \in \{a, b\}^*\). Нека \(|\alpha| = n + 1\).

Нека тогава \(y, z \in \mathbb{N}\) и нека \(\rho, \xi \in \Sigma^*\) са такива, че \(\rho.y = \alpha = z.\xi\).

Така \(\alpha.u.\beta\#\alpha.flip(u).\gamma = (z.\xi).u.\beta\#(\rho.y).flip(u).\gamma\).

Имаме \(|\xi| = |\rho|\) и значи \(\xi.u.\beta\# \in \mathcal{CFL}_G(var(u))\).

Нека \(V = var(u)\) и нека \(m \in \mathbb{N}\) е такова, че \(V \overset{m}{\leadsto} \xi.u.\beta\#\rho\).

Имаме правило \(V \to XVX\) и значи \(V \overset{1 + max(1, m, 1)}{\leadsto} z\xi.u.\beta\#\rho.y\).

Имаме \(\gamma \in \{a, b\}^* = \mathcal{CFL}_G(F)\). Нека тогава \(t \in \mathbb{N}\) е такова, че \(F \overset{t}{\leadsto} \gamma\).

Имаме правило \(S \to V . flip(u) . F\). Нека \(h = 1 + max(1 + max(1, m), t)\).

Така \(S \overset{h}{\leadsto} \alpha.u.\beta\#\alpha.flip(u).\gamma = (z.\xi).u.\beta\#(\rho.y).flip(u).\gamma\).

Следователно \(\alpha.u.\beta\#\alpha.flip(u).\gamma \in \mathcal{CFL}_G(S)\).

\subsection*{Заключение.}
От индукцията получаваме 
\((\forall \omega \in L)(\omega \in \mathcal{CFL}_G(S))\).

Следователно \(L \subseteq \mathcal{CFL}_G(S)\). Следователно \(\mathcal{CFL}_G(S) = L\).

Следователно \(\mathcal{CFL}(G) = L\) и значи \(L\) е безконтекстен.

\end{document}
