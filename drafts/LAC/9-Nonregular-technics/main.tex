\documentclass[12pt]{article}
\usepackage[utf8]{inputenc}
\usepackage[T2A]{fontenc}
\usepackage[english,bulgarian]{babel}
\usepackage{amsmath}
\usepackage{amssymb}
\usepackage{amsthm}
\usepackage{euler}

\title{Две техники за доказване на нерегулярност}
\author{Иво Стратев}

\begin{document}

\maketitle

\section*{Техника 1: Сечение с регулярен}

Нека \(L = \{a^i.b^j.c^k \;\mid\; i \in \mathbb{N} \;\&\; j \in \mathbb{N} \;\&\; k \in \mathbb{N} \;\&\; i = j \;\lor\; j = k\}\).
Да допуснем, че \(L\) е регулярен.
Тогава е регулярен и \(L \cap \{a\}^*\cdot\{b\}^*\).
Но \(L \cap \{a\}^*\cdot\{b\}^* = \{a^n.b^n.c^0 \;\mid\; n \in \mathbb{N}\} = \{a^n.b^n \;\mid\; n \in \mathbb{N}\}\).
Както знаем обаче езикът \(\{a^n.b^n \;\mid\; n \in \mathbb{N}\}\) не е регулярен.
Достигнахме до Абсурд!
Следователно \(L\) не е регулярен.

\section*{Техника 2: Допълнение и сечение с регулярен}
Нека \(L = \{a^i.b^j.c^k \;\mid\; i \in \mathbb{N} \;\&\; j \in \mathbb{N} \;\&\; k \in \mathbb{N} \;\&\; i \neq j \;\lor\; j \neq k \;\lor\; i \neq k\}\).
Да допуснем, че \(L\) е регулярен.
\(L\) е език над азбуката \(\{a, b, c\}\).
Тогава допълнението на \(L\) е \(\{a, b, c\}^* \setminus L\), което е регулярен език,
защото автоматните и регулрните езици съвпадат, а автоматните са затворени относно допълнение.

\vspace*{5mm}

\par Тогава е регулярен и \((\{a, b, c\}^* \setminus L) \cap (\{a\}^*\cdot\{b\}^*\cdot\{c\}^*)\).
Но това значи, че \(\{a^n.b^n.c^n \;\mid\; n \in \mathbb{N}\}\) е регулярен език,
а той не е (помислете защо).
Достигнахме до Абсурд!
Следователно \(L\) не е регулярен.

\vspace*{5mm}

\par \textbf{Забележка}: В случая не можем да използваме само сечение с \\
\(\{a\}^*\cdot\{b\}^*\cdot\{c\}^*\),
защото сечението съвпада с \(L\). Но не можем да използваме и само допълнение, защото допълнението на \(L\) е строго надмножество на езика \(\{a^i.b^j.c^k \;\mid\; i \in \mathbb{N} \;\&\; j \in \mathbb{N} \;\&\; k \in \mathbb{N} \;\&\; i = j \;\&\; j = k \;\&\; i = k\}\). Например в него се съдържат думите \(ba\), \(babc\), \(cabcba\) и тн.
Налага да пресечем допълнението с \(\{a\}^*\cdot\{b\}^*\cdot\{c\}^*\), за да фиксираме реда на буквите в думите.

\end{document}