\documentclass[14pt]{extarticle}
\usepackage{tikz}
\usepackage{tikz-qtree}
\usepackage{float}
\usepackage[utf8]{inputenc}
\usepackage[T2A]{fontenc}
\usepackage[english,bulgarian]{babel}
\usepackage{amsmath}
\usepackage{amssymb}
\usepackage{amsthm}
\usepackage{relsize}
\usepackage{euler}
\usepackage[a4paper, portrait, margin=1.8cm]{geometry}

\title{Еднолентови детерминистични машини на Тюринг}
\author{Иво Стратев}

\begin{document}

\maketitle

\section*{Дефиниция за еднолентова машина на Тюринг}
\(\langle Q,\; \Sigma,\; \Gamma,\; \delta,\; s,\; H \rangle\) наричаме \textbf{еднолентова машина на Тюринг}, ако

\begin{itemize}
    \item \(Q\) е крайно множество. Множеството \(Q\) е множеството на контролните състояния на машината.
    \item \(\Sigma\) е крайна азбука, такава че \(\Sigma \cap \{\triangleright, \sqcup, \leftarrow, \rightarrow\} = \emptyset\). Азбуката \(\Sigma\) е входната азбука на машината.
    \item \(\Gamma\) е крайна множество, такова че \(\Sigma \cap \Gamma = \emptyset\) и \(\Gamma \cap \{\triangleright, \sqcup, \leftarrow, \rightarrow\} = \emptyset\). Множеството \(\Gamma\) е множеството на спомагателните символи за машината.
    \item \(s \in Q\). \(s\) е началното състояние.
    \item \(H\) е крайно множество, такова че \(Q \cap H = \emptyset\). Множеството \(H\) е множеството на стоп състоянията.
    \item \(\delta : Q \times (\Sigma \cup \Gamma \cup \{ \triangleright, \sqcup \}) \to H \cup (Q \times (\Sigma \cup \Gamma \cup \{\leftarrow, \rightarrow\}))\). \(\delta\) е контролната функция. която задава програмата на машината и \(\delta\) има свойството
    \[(\forall p, q \in Q)(\forall x \in \Sigma \cup \Gamma \cup \{\leftarrow, \rightarrow\})(\delta(p, \triangleright) = \langle q, x \rangle \implies x = \;\rightarrow )\]
\end{itemize}


\subsection*{Програма на еднолентова машина на Тюринг}
Нека \(\langle Q,\; \Sigma,\; \Gamma,\; \delta,\; s,\; H \rangle\) e еднолентова машина на Тюринг.


\subsubsection*{Оператори за четене и промяна на буква}
Въвеждаме помощни оператори за четене от дума и промяна на буква
\begin{align*}
    (.)[.] : \{\triangleright\} \cdot (\Sigma \cup \Gamma \cup \{\sqcup\})^* \times \mathbb{N}_+ \to \Sigma \cup \Gamma \cup \{\sqcup, \triangleright\}  \\
    (.)[.,\; .] : \{\triangleright\} \cdot (\Sigma \cup \Gamma \cup \{\sqcup\})^* \times (\mathbb{N} \setminus \{0, 1\} ) \times (\Sigma \cup \Gamma) \to \{\triangleright\} \cdot (\Sigma \cup \Gamma \cup \{\sqcup\})^* 
\end{align*}

\begin{align*}
    \alpha[i] = \begin{cases}
        \alpha_i &,\; i \leq |\alpha| \\
        \sqcup &,\; i > |\alpha|
    \end{cases}
\end{align*}

\begin{align*}
    (\triangleright\alpha)[i, x] = \triangleright. (\alpha \;!\; \langle i - 1, x \rangle) \quad where \\
    \varepsilon \;!\; \langle k + 1, z \rangle = \sqcup^{k} z \\
    (y.\sigma) \;!\; \langle 1, z \rangle = z.\sigma \\
    (y.\sigma) \;!\; \langle k + 2, z \rangle = y.(\sigma \;!\; \langle k + 1, z \rangle)
\end{align*}

Нека пресметнем два примера

\begin{align*}
    \triangleright ab [6, c] = \\
    \triangleright . (ab \;!\; \langle 5, c \rangle) = \\
    \triangleright . (a . (b \;!\; \langle 4, c \rangle)) = \\
    \triangleright . (a . (b . (\varepsilon \;!\; \langle 3, c \rangle))) = \\
    \triangleright . (a . (b . (\sqcup^2 c))) = \\
    \triangleright ab \sqcup \sqcup c
\end{align*}
\begin{align*}
    \triangleright abcc [3, d] = \\
    \triangleright . (abcc \;!\; \langle 2, d \rangle) = \\
    \triangleright . (a . (bcc \;!\; \langle 1, d \rangle)) = \\
    \triangleright . (a . (d.cc)) = \\
    \triangleright adcc
\end{align*}

\subsubsection*{Стъпкови програми}

За всяко естествено число \(n\) въвеждаме функция 
\(\delta^n\)

от \(Q \times (\{\triangleright\} \cdot (\Sigma \cup \Gamma \cup \{\sqcup\})^* ) \times \mathbb{N}_+\)
в \((Q \cup H) \times (\{\triangleright\} \cdot (\Sigma \cup \Gamma \cup \{\sqcup\})^* ) \),

която описва поведението на машината при направата на неповече от \(n\) стъпки
чрез рекурсия така
\begin{align*}
    \delta^0(q, \alpha, i) = \langle q, \alpha \rangle
\end{align*}

Ако \(n \in \mathbb N\), \(q \in Q\), \(\alpha \in \{\triangleright\} \cdot (\Sigma \cup \Gamma \cup \{\sqcup\})^* \), \(i \in \mathbb{N}_+\), то

\begin{itemize}
    \item ако \(\delta(q, \alpha[i]) \in H\), то \(\delta^{n + 1}(q, \alpha, i) = \langle \delta(q, \alpha[i]), \alpha \rangle \) (ако попаднем в стоп състояние прекратяваме изчислението);
    \item ако \(\langle t, x \rangle = \delta(q, \alpha[i])\) и \(x \in \Sigma \cup \Gamma\), то \(\delta^{n + 1}(q, \alpha, i) = \delta^n(t, \alpha[i, x], i)\) (ако получим команда за промяна на буквата на поция \(i\) извършваме тази команда, като ако се налага разшираяваме думата добавяйки необходимият брой шпации между думата и символа \(x\));
    \item ако \(\langle t, \rightarrow \rangle = \delta(q, \alpha[i])\), то \(\delta^{n + 1}(q, \alpha, i) = \delta^n(t, \alpha, i + 1)\) (ако получим команда за отместване на курсора една позиция на дясно, то само местим курсора на дясно);
    \item ако \(\langle t, \leftarrow \rangle = \delta(q, \alpha[i])\), то \(\delta^{n + 1}(q, \alpha, i) = \delta^n(t, \alpha, i - 1)\) (ако получим команда за отместване на курсора една позиция на ляво, то само местим курсора на ляво, понеже \(\alpha\) започва с \(\triangleright\), то \(i > 1\) и значи \(i - 1 \in \mathbb{N}_+\)).
\end{itemize}

Нека означим езика \(\{\triangleright\} \cdot (\Sigma \cup \Gamma \cup \{\sqcup\})^*\) с \(\Lambda\). В сила е следното твърдение
\begin{align*}
    (\forall q \in Q)(\forall \alpha \in \Lambda)(\forall i \in \mathbb{N}_+)(\exists n \in \mathbb N)(\exists h \in H)(\exists \beta \in \Lambda)\\
    (\delta^n(q, \alpha, i) = \langle h, \beta \rangle \implies (\forall k \in \mathbb N)(k \geq n \implies \delta^k(q, \alpha, i) = \langle h, \beta \rangle) )
\end{align*}

Защото е в сила следното твърдение
\begin{align*}
    (\forall q, t \in Q)(\forall \alpha, \beta \in \Lambda)(\forall i, j \in \mathbb{N}_+)(\forall n \in \mathbb N)(\forall k \in \{0, 1, \dots, n\})\\(
        \delta^n(q, \alpha, i) = \delta^k(t, \beta, j) \implies (\forall s \in \mathbb N)(\delta^{n + s}(q, \alpha, i) = \delta^{k + s}(t, \beta, j))
    )
\end{align*}
Също така в сила е
\begin{align*}
    (\forall q \in Q)(\forall \alpha, \beta \in \Lambda)(\forall i \in \mathbb{N}_+)(\forall n \in \mathbb N)(\forall h \in H)\\(
        \delta^n(q, \alpha, i) = \langle h, \beta \rangle \implies (\exists k \in \{1, \dots, n\})(\delta^k(q, \alpha, i) = \langle h, \beta \rangle)
    )
\end{align*}

Да разгледаме два примера. \\

\textbf{Пример 1.} Нека разгледаме следната еднолентова машина на Тюринг:

\(\langle \{s, m, e\},\; \{0, 1\},\; \emptyset,\; \delta,\; s,\; \{h, f\} \rangle\), където


\begin{align*}
    \delta : \{s, m, e\} \times \{0, 1, \triangleright, \sqcup\} \to \{h, f\} \cup (\{s, m, e\} \times \{0, 1, \rightarrow, \leftarrow \}) \\
    \delta(s, 0) = f \\
    \delta(s, 1) = f \\
    \delta(s, \triangleright) = f \\
    \delta(s, \sqcup) = \langle m, 1 \rangle \\
    \delta(m, 0) = \langle m, \rightarrow \rangle \\
    \delta(m, 1) = \langle m, \rightarrow \rangle \\
    \delta(m, \triangleright) = f \\
    \delta(m, \sqcup) = \langle e, 1 \rangle \\
    \delta(e, 0) = f \\
    \delta(e, 1) = h \\
    \delta(e, \triangleright) = f \\
    \delta(e, \sqcup) = f
\end{align*}
Тогава
\begin{align*}
    \delta^2(s, \triangleright \underline{\sqcup} 10, 2) =  \quad (by \; \delta(s, \sqcup) = \langle m, 1 \rangle)\\
    \delta^1(m, \triangleright \underline{1} 10, 2) = \quad (by \; \delta(m, 1) = \langle m, \rightarrow \rangle) \\
    \delta^0(m, \triangleright 1 \underline{1} 0, 3) =  \quad (by \; def \; of \; \delta^0) \\
    \langle m, \; \triangleright 1 10 \rangle
\end{align*}
Също така
\begin{align*}
    \delta^9(s, \triangleright \underline{\sqcup} 10, 2) =  \quad (by \; \delta(s, \sqcup) = \langle m, 1 \rangle)\\
    \delta^8(m, \triangleright \underline{1} 10, 2) = \quad (by \; \delta(m, 1) = \langle m, \rightarrow \rangle) \\
    \delta^7(m, \triangleright 1 \underline{1} 0, 3) =  \quad (by \; \delta(m, 1) = \langle m, \rightarrow \rangle) \\
    \delta^6(m, \triangleright 1 1 \underline{0}, 4) =  \quad (by \; \delta(m, 0) = \langle m, \rightarrow \rangle) \\
    \delta^5(m, \triangleright 1 1 0 \_, 5) =  \quad (by \; \delta(m, \sqcup) = \langle e, 1 \rangle) \\
    \delta^4(e, \triangleright 1 1 0 \underline{1}, 5) =  \quad (by \; \delta(e, 1) = h) \\
    \langle h, \triangleright 1 1 01\rangle
\end{align*}
Очевидно \(\delta^6(s, \triangleright \sqcup 10, 2) = \langle h,\; \triangleright 1 1 01\rangle\). \\

\textbf{Пример 2.} Нека разгледаме следната еднолентова машина на Тюринг:

\(\langle \{s, m\},\; \{0, 1\},\; \emptyset,\; \delta,\; s,\; \{h, f\} \rangle\), където
\begin{align*}
    \delta : \{s, m\} \times \{0, 1, \triangleright, \sqcup\} \to \{h, f\} \cup (\{s, m\} \times \{0, 1, \rightarrow, \leftarrow \}) \\
    \delta(s, 0) = \langle m, 1 \rangle \\
    \delta(s, 1) = \langle m, 0 \rangle \\
    \delta(s, \triangleright) = f \\
    \delta(s, \sqcup) = h \\
    \delta(m, 0) = \langle s, \rightarrow \rangle \\
    \delta(m, 1) = \langle s, \rightarrow \rangle \\
    \delta(m, \triangleright) = f \\
    \delta(m, \sqcup) =  f 
\end{align*}

Тогава
\begin{align*}
    \delta^1(s, (\triangleright \varepsilon)_, 2) = \langle h, \triangleright\varepsilon \rangle  \quad (by \; \delta(s, \sqcup) = h \rangle)
\end{align*}
\begin{align*}
    \delta^7(s, \triangleright \underline{1}01, 2) =  \quad (by \; \delta(s, 1) = \langle m, 0 \rangle \rangle) \\
    \delta^6(m, \triangleright \underline{0}01, 2) =  \quad (by \; \delta(m, 0) = \langle s, \rightarrow \rangle \rangle) \\
    \delta^5(s, \triangleright 0\underline{0}1, 3) =  \quad (by \; \delta(s, 0) = \langle m, 1 \rangle \rangle) \\
    \delta^4(m, \triangleright 0\underline{1}1, 3) =  \quad (by \; \delta(m, 1) = \langle s, \rightarrow \rangle \rangle) \\
    \delta^3(s, \triangleright 01\underline{1}, 4) =  \quad (by \; \delta(s, 1) = \langle m, 0 \rangle \rangle) \\
    \delta^2(m, \triangleright 01\underline{0}, 4) =  \quad (by \; \delta(m, 0) = \langle s, \rightarrow \rangle \rangle) \\
    \delta^1(s, \triangleright 010\_, 5) =  \quad (by \; \delta(m, \sqcup) = h \rangle) \\
    \langle h, \triangleright 010 \rangle
\end{align*}
Така получихме \(\delta^7(s, \triangleright 101, 2) = \langle h,\; \triangleright 010 \rangle\).

\subsubsection*{Програма на еднолентова машина на Тюринг}
Нека \(\mathcal M = \langle Q,\; \Sigma,\; \Gamma,\; \delta,\; s,\; H \rangle\) e еднолентова машина на Тюринг.

Нека \(k \in \mathbb N\). Дефинираме частична функция \(\Pi_k^{\mathcal M}\) от \(\Sigma^*\) в

\(H \times (\{\triangleright\} \cdot (\Sigma \cup \Gamma \cup \{\sqcup\})^* ) \) като

\[\{ \langle \alpha, r \rangle \mid \alpha \in \Sigma^* \;\&\; r \in H \times (\{\triangleright\} \cdot (\Sigma \cup \Gamma \cup \{\sqcup\})^* ) \;\&\; (\exists n \in \mathbb{N}_+)(r = \delta^n(s,\; \triangleright \sqcup^k \alpha,\; 2))  \} \]

Няма да доказваме, че така дефинирана \(\Pi_k^{\mathcal M}\) в действителност е функция макар и частична.
Реално \(\Pi_k^{\mathcal M}\) по дума \(\alpha \in \Sigma^*\) ако има \(n \in \mathbb N\), такова, че \(n\)-тата стъпкова програма по машината \(\mathcal M\) завършва (изпада в стоп състояние) при стартиране на изчисление от началното състояние, при инциализирана лента, на която между възпиращият символ \(\triangleright\) и думата \(\alpha\) има предварително заделена памет от \(k\) празни клетки (blank символи) и курсор сетнат на позиция \(2\) \;"връща"\; изчислението направено от \(\delta^n\). \\

Например ако разглеждаме машината от \textbf{Пример 1}, то \(\Pi_1\) със сигурност е дефинирана в \(10\) и при това
\(\Pi_1(10) = \langle h, \triangleright 1101\rangle\). Дори още по-силно може да се докаже, че в този пример \(\Pi_1\) е тотална функция от \(\{0, 1\}^*\) в \(\{h\} \times (\{\triangleright 1\} \cdot \{0, 1\}^* \cdot \{1\})\) действаща по правилото \(\Pi_1(\alpha) = \langle h,\; \triangleright.1.\alpha.1 \rangle\). \\

Ако разгледаме машината от \textbf{Пример 2}, то може да се докаже, че \(\Pi_0\) е тотална функция с домейн \(\{0, 1\}^*\) действаща по правилото
\(\Pi_0(\alpha) = \langle h, \triangleright . \overline{\alpha} \rangle\),
където ако \(\alpha \in \{0, 1\}^*\), то
\(\overline{\alpha}\) е означение за думата, която се получава от \(\alpha\) при побуквено инвертиране.
Например от направеното пресмятане \(\Pi_0(101) = \langle h,\; \triangleright010\rangle\). \\

\subsection*{Частичната функция, която изчислява дадена ЕМТ спрямо фиксирано стоп състояние}
Нека \(\mathcal M = \langle Q,\; \Sigma,\; \Gamma,\; \delta,\; s,\; H \rangle\) e еднолентова машина на Тюринг.
Нека \(h \in H\). Нека \(k \in \mathbb N\). Частичната функция, която \(M\) изчислява с начална буферна памет от \(k\) клетки спрямо стоп състояние \(h\) ще бележим с \([\mathcal M, h, k]\) и тя е 
\[\{ \langle \alpha, \beta \rangle \in \Sigma^* \times (\Sigma \cup \Gamma)^* \mid \langle \alpha, \langle h, \triangleright\beta \rangle \rangle \in \Pi_k^{\mathcal M}  \}\]

Например ако разглеждаме машината от \textbf{Пример 1}, то

\[[\mathcal M, h, 1] = \{ \langle \alpha, 1 \alpha 1 \rangle \mid \alpha \in \{0, 1\}^*  \} \]

Тоест \([\mathcal M, h, 1]\) е тотална функция и действа по правилото
\[ [\mathcal M, h, 1](\alpha) = 1\alpha1 \]

Ако разглеждаме машината от \textbf{Пример 2}, то

\[[\mathcal M, h, 0] = \{ \langle \alpha, \overline{\alpha} \rangle \mid \alpha \in \{0, 1\}^*  \} \]

Тоест \([\mathcal M, h, 0]\) е тотална функция и действа по правилото
\[ [\mathcal M, h, 0](\alpha) = \overline{\alpha} \]

\subsection*{Дефиниция за изчислима функция между азбуки}
Нека \(I\) и \(O\) са две крайни азбуки.
Нека \(f : I^* \to O^*\). 
Казваме, че \(f\) е изчислима функция между азбуки, ако
съществува ЕМТ \(\mathcal M = \langle Q,\; \Sigma,\; \Gamma,\; \delta,\; s,\; H \rangle\),
съществува \(h \in H\) и съществува \(k \in \mathbb N\) такива, че \(f \subseteq [\mathcal M, h, k]\). \\

Ако \(\mathcal M = \langle Q,\; \Sigma,\; \Gamma,\; \delta,\; s,\; H \rangle\), \(h \in H\) и \(k \in \mathbb N\)
са такива, че \(f \subseteq [\mathcal M, h, k]\), то ясно е, че \(I \subseteq \Sigma\) и \(O \subseteq \Sigma \cup \Gamma\).

\subsection*{Дефиниция за изчислима функция между езици}
Нека \(L\) е език над крайна азбука \(I\) и \(K\) е език над крайна азбука \(O\).
Нека \(f : L \to K\). Казваме, че \(f\) е изчислима функция между езици, ако
съществува ЕМТ \(\mathcal M = \langle Q,\; \Sigma,\; \Gamma,\; \delta,\; s,\; H \rangle\),
съществува \(h \in H\) и съществува \(k \in \mathbb N\) такива, че \(f \subseteq [\mathcal M, h, k]\).

\subsection*{Дефиниция за изчислима \(s\)-местна частична функция между естествени числа}
Нека от сега до края с \(Bin\) означим множеството на всички правилни двоични записи на естествени числа.
Тоест \(Bin = \{0\} \cup \{1\} \cdot \{0, 1\}^*\). Нека с \(B\) означим биекцията между \(\mathbb N\) и \(Bin\),
която на всяко естествено число съпоставя думата, която е неговият двоичен запис.
Например \(B(5) = 101\). \\

Нека \(s \in \mathbb N\) и \(f\) е частична фунцкия между \(\mathbb{N}^s\) и \(\mathbb N\).
Ще казваме, че \(f\) е изчислима \(s\)-местна частична функция между естествени числа, ако
\[ \{ \langle \; B(x_1) \$ B(x_2) \$ \dots \$ B(x_s) \; , \quad B( f(x_1, x_2, \dots, x_s) ) \;  \rangle \mid \langle x_1, x_2, \dots, x_s \rangle \in \mathbb N^s  \} \]
е изчислима функция между езиците \( Bin \cdot (\{\$\} \cdot Bin)^{s - 1} \) и \(Bin\). \\

Например тоталната функция \(g : \mathbb N \to \mathbb N\) действаща по правилото
\[ g(n) = \Large{2^{Log(n) + 2} + 2n + 1 } \]
е изчислима, защото фунцкията \(h : Bin \to Bin\) действаща по правилото \(h(\alpha) = 1.\alpha.1\) е изчислима от машината от \textbf{Пример 1}, където
\[ Log(n) = \begin{cases}
    0 &,\; n = 0 \\
    \lfloor log_2(n) \rfloor &,\; n \geq 1
\end{cases} \]

\subsection*{Разрешимост}
До сега се интересувахме от стоп състоянието, в което изпада една ЕМТ както и каква дума остава на лентата след конкретен вход на лентата. Интересувайки се от думата, която остава на лентата гледахме на една ЕМТ като на \textbf{генератор}. \\

\par Можем да забравим за резултата на лентата, а да се интересуваме само от стоп състоянието, в което изпада машината при конкретен вход. Така ще гледаме на една ЕМТ като на \textbf{разпознавател}, ако тя има само две стоп състояния.

\subsubsection*{Частичната функция \(halt\)}
Нека \(\mathcal M = \langle Q,\; \Sigma,\; \Gamma,\; \delta,\; s,\; H \rangle\) е ЕМТ.

Дефинираме частична функция \(halt_{\mathcal M}\) от \(\Sigma^*\) в \(H\) така

\[\{ \langle \alpha, h \rangle \in \Sigma^* \times H \mid (\exists \beta \in \{\triangleright\} \cdot (\Sigma \cup \Gamma \cup \{\sqcup\})^*) ( \langle \alpha, \langle h, \beta \rangle \rangle \in \Pi_1^{\mathcal M})  \}\]

\subsubsection*{Дефиниция за разрешимимост от ЕМТ на език}
Нека \(\Sigma\) е крайна азбука.
Нека \(L\) е език над \(\Sigma\).
Нека \(\mathcal M = \langle Q,\; \Sigma,\; \Gamma,\; \delta,\; s,\; \{no, yes\} \rangle\) е ЕМТ.
Казваме, че \(\mathcal M\) разрешава \(L\), ако са изпълнени следните три условия

\begin{itemize}
    \item \(halt_{\mathcal M}\) е тотална функция;
    \item \((\forall \omega \in L)( halt_{\mathcal M}(\omega) = yes )\);
    \item \((\forall \omega \in \Sigma^* \setminus L)( halt_{\mathcal M}(\omega) = no )\).
\end{itemize}

\subsubsection*{Дефиниция за разрешим език}
Нека \(\Sigma\) е крайна азбука.
Нека \(L\) е език над \(\Sigma\).
Казваме, че \(L\) е разрешим език, ако \(L\) се разрешава от някоя ЕМТ.
Тоест \(L\) е разрешим език ТСТК съществува ЕМТ \(\mathcal M = \langle Q,\; \Sigma,\; \Gamma,\; \delta,\; s,\; \{no, yes\} \rangle\) такава, че \(\mathcal M\) разрешава \(L\).

\subsubsection*{Дефиниция за полуразрешимимост от ЕМТ на език}
Нека \(\Sigma\) е крайна азбука.
Нека \(L\) е език над \(\Sigma\).
Нека \(\mathcal M = \langle Q,\; \Sigma,\; \Gamma,\; \delta,\; s,\; \{no, yes\} \rangle\) е ЕМТ.
Казваме, че \(\mathcal M\) полуразрешава \(L\), ако са изпълнени следните три условия

\begin{itemize}
    \item \(L \subseteq Dom(halt_{\mathcal M})\) т.е \(halt_{\mathcal M}\) е дефинирана върху всяка дума от \(L\);
    \item \((\forall \omega \in L)( halt_{\mathcal M}(\omega) = yes )\);
    \item \((\forall \omega \in \Sigma^* \setminus L)(\omega \in Dom(halt_{\mathcal M}) \implies  halt_{\mathcal M}(\omega) = no )\).
\end{itemize}

\subsubsection*{Дефиниция за полуразрешим език}
Нека \(\Sigma\) е крайна азбука.
Нека \(L\) е език над \(\Sigma\).
Казваме, че \(L\) е полуразрешим език, ако \(L\) се полуразрешава от някоя ЕМТ.
Тоест \(L\) е разрешим език ТСТК съществува ЕМТ \(\mathcal M = \langle Q,\; \Sigma,\; \Gamma,\; \delta,\; s,\; \{no, yes\} \rangle\) такава, че \(\mathcal M\) полуразрешава \(L\).

\subsubsection*{Важна теорема}
Нека \(\Sigma\) е крайна азбука.
Нека \(L\) е език над \(\Sigma\).
Очевидно ако \(L\) е разрешим, то \(L\) е полуразрешим.
Обратното не е вярно. Известно е, че \textbf{СТОП}-проблемът не е разрешим, но е полуразрешим! \\

\textbf{Теорема.} \(L\) е разрешим ТСТК \(L\) и \(\Sigma^* \setminus L\) са полуразрешими езици.

\subsubsection*{Пример за разрешим език}
Нека \(\Sigma\) е крайна азбука.
Нека \(L = \{\omega\#\omega \mid \omega \in \Sigma^+\}\).

Ще докажем, че \(L\) е разрешим като построим ЕМТ, която го разрешава.

Нека разгледаме няколко примера за да хванем как бихме могли да разрешим \(L\).
Нека за примерите да приемем, че \(\Sigma = \{a, b, c\}\). \\

Първо добре е да се подсигурим, че \(\#\) разделя две думи с положителна дължина.
За целта трябва да прочетем цялата активна лента.
\begin{align*}
    \triangleright \underline{\sqcup} abc \# abc \\
    \triangleright \sqcup abc \# abc\_
\end{align*}
След това да се върнем назад и да запомним видяният символ и да го изтрием.
След, което да вървим наляво преминавайки през \(\#\) до първата срещната буква от \(\Sigma\).
\begin{align*}
    \triangleright \sqcup abc \# ab\underline{c} \\
    c: \; \triangleright \sqcup abc \# ab\underline{\$} \\
    c: \; \triangleright \sqcup abc \# a\underline{b}\$ \\
    c: \; \triangleright \sqcup abc \# ab\underline{\$} \\
    c: \; \triangleright \sqcup abc \underline{\#} ab\$
\end{align*}
Когато я стигнем трябва да я сравним със запомнената, ако съвпада да я изтрием и да върви надясно до първият \(\$\).
Защото след него или следват други \(\$\) или активната част свършва.
\begin{align*}
    c: \; \triangleright \sqcup ab\underline{c} \# ab\$ \\
    \triangleright \sqcup ab\underline{\$} \# ab\$ \\
    \triangleright \sqcup ab\$ \underline{\#}ab\$ \\
    \triangleright \sqcup ab\$ \# ab\underline{\$}
\end{align*}
След като стигнем \(\$\) се връщаме назад и повтаряме процедурата.
\begin{align*}
    \triangleright \sqcup ab\$ \# a\underline{b}\$ \\
    b: \; \triangleright \sqcup ab\$ \# a\underline{\$}\$ \\
    b: \; \triangleright \sqcup ab\$ \# \underline{a}\$\$ \\
    b: \; \triangleright \sqcup ab\$ \underline{\#} a\$\$ \\
    b: \; \triangleright \sqcup ab\underline{\$}\# a\$\$ \\
    b: \; \triangleright \sqcup a\underline{b}\$\# a\$\$ \\
    \triangleright \sqcup a\underline{\$}\$ \# a\$\$ \\
    \triangleright \sqcup a\$\$ \underline{\#}a\$\$ \\
    \triangleright \sqcup a\$\$ \# a\underline{\$}\$ \\
    \triangleright \sqcup a\$\$ \# \underline{a}\$\$ \\
    a: \; \triangleright \sqcup a\$\$ \# \underline{\$}\$\$ \\
    a: \; \triangleright \sqcup a\underline{\$}\$ \# \$\$\$ \\
    a: \; \triangleright \sqcup \underline{a}\$\$ \# \$\$\$ \\
    \triangleright \sqcup \underline{\$}\$\$ \# \$\$\$ \\
\end{align*}
След изтриването отново се движим надясно до достигането на първия \(\$\).
\begin{align*}
    \triangleright \sqcup \$\$\$ \# \underline{\$}\$\$ \\
    \triangleright \sqcup \$\$\$ \underline{\#} \$\$\$ \\
\end{align*}
Връщайки се назад обаче вместо буква от \(\Sigma\) виждаме \(\#\),
което значи, че трябва да започнем да се движим наляво и да достигнем до \(\sqcup\) срещайки само \(\$\). В тази ситуация приемаме думата.
Ако обаче видим буква от \(\Sigma\) трябва да отхвърлим, защото лявата дума е с дължина строго по-голяма от дясната и няма как да са равни! Така горе-долу знаем какво да правим ако лявата дума съвпада с дясната или е по-дълга или суфиксът ѝ с дължина дължината на дясната дума не съвпада!
Остава да разгледаме ситуацията, когато дясната дума е строго по-дълга от лявата.
\begin{align*}
    \triangleright \sqcup abc \# aabc \\
    \triangleright \sqcup abc \# aab\underline{c} \\
    \vdots \\
    \triangleright \sqcup \$\$\$ \# a\underline{\$}\$\$ \\
    \triangleright \sqcup \$\$\$ \# \underline{a}\$\$\$ \\
    a: \; \triangleright \sqcup \$\$\$ \# \underline{\$}\$\$\$ \\
    a: \; \triangleright \sqcup \underline{\$}\$\$ \# \$\$\$\$ \\
    a: \; \triangleright \underline{\sqcup}\$\$\$ \# \$\$\$\$
\end{align*}
Оказва се, че в тази ситуация трябва да сме във състояние, в което сме запомнили конкретна буква от \(\Sigma\), но след последният \(\$\) от ляво не виждаме запомнената буква, а виждаме \(\sqcup\) в тази ситауция отхвърляме! \\

Стъпка по стъпка ще направим ЕМТ \(\langle Q,\; \Sigma \cup \{\#\},\; \Gamma,\; \delta,\; start,\; \{no, yes\} \rangle\),
която да разреши \(L\) в обща ситуация, в която реално не знаем \(\Sigma\).
Първо от нашите разсъждения е ясно, че ще ни трябва само \(\$\) за помощен символ, следователно \(\Gamma = \{\$\}\).
Постепенно ще правим преходите, така че директно ще получим \(\delta\). Реално остава само \(Q\). \\

Ще даваме само положитлните преходи, всички други, които са необходими за да стане \(\delta\) тотална функция ще бъдат \(no\).
Започваме от преходите, с които да прочетем активната част на лентата и да се уверим, че тя е от вида \(\triangleright\sqcup\alpha\#\beta\), където \(|\alpha| > 0\) и \(|\beta| > 0\).
\begin{align*}
    \delta(start, \sqcup) = \langle left, \rightarrow \rangle \\
    (\forall x \in \Sigma)(\delta(left, x) = \langle mid, \rightarrow \rangle)\\
    \delta(mid, \#) = \langle right, \rightarrow \rangle \\
    (\forall x \in \Sigma)(\delta(mid, x) = \langle mid, \rightarrow \rangle) \\
    (\forall x \in \Sigma)(\delta(right, x) = \langle seek, \rightarrow \rangle) \\
    \delta(seek, \sqcup) = \langle read, \leftarrow \rangle \\
    (\forall x \in \Sigma)(\delta(seek, x) = \langle seek, \rightarrow \rangle)
\end{align*}
С горните преходи си подсигуряваме точно вида на лентата, който ни трябва на първо четене и гарантирано, че ще сме в състояние \(read\). В състояние \(read\), трябва да прочетем и запомним буква от дясно на \(\#\).
След, което да минем наляво пропускай всички букви отдясно, \(\#\) и евентуалните \(\$\) от ляво.
Като ясно е, че по някакъв начин като минем през \(\#\) трябва да знаем, че трябва да минем във фаза matching.
Тоест не трябва да преминаваме през символите на \(\Sigma\), а като стигнем до първия да го сравним със запомненият.
\begin{align*}
    (\forall x \in \Sigma)(\delta(read, x) = \langle read_x, \$ \rangle) \\
    (\forall x \in \Sigma)(\delta(read_x, \$) = \langle read_x, \leftarrow \rangle) \\
    (\forall x \in \Sigma)(\forall y \in \Sigma)(\delta(read_x, y) = \langle read_x, \leftarrow \rangle)\\
    (\forall x \in \Sigma)(\delta(read_x, \#) = \langle match_x, \leftarrow \rangle \\
    (\forall x \in \Sigma)(\delta(match_x, \$) = \langle match_x, \leftarrow \rangle \\
    (\forall x \in \Sigma)(\delta(match_x, x) = \langle cross, \$ \rangle)
\end{align*}
След като успешно match-нем буквите изтриваме, заменяйки с \(\$\).
След това, трябва да вървим надясно и да пресечем \(\#\) където имаме два варианта:
да видим буква или \(\$\). Ако видим \(\$\) значи сме изчерпали думата от дясно и трябва да вървим на ляво и за да приемем да видим \(\$\) последниван/и от \(\sqcup\). Ако видим буква от \(\Sigma\) трябва да направим поне още една итерация.
\begin{align*}
    \delta(cross, \$) = \langle cross, \rightarrow \rangle \\
    \delta(cross, \#) = \langle scout, \rightarrow \rangle \\
    \delta(scout, \$) = \langle finish, \leftarrow \rangle \\
    (\forall x \in \Sigma)(\delta(scout, x) =  \langle seek, \rightarrow \rangle ) \\
    \delta(seek, \$) = \langle read, \leftarrow \rangle \\
    \delta(finish, \#) = \langle finish, \leftarrow \rangle \\
    \delta(finish, \$) = \langle finish, \leftarrow \rangle \\
    \delta(finish, \sqcup) = yes
\end{align*}
Нека видим, че думата \(ab \# ab\) ще бъде приета.
\begin{align*}
    \delta^{33}(start, \triangleright \underline{\sqcup} ab \# ab, 2) = \quad (by \; \delta(start, \sqcup) = \langle left, \rightarrow \rangle) \\
    \delta^{32}(left, \triangleright \sqcup \underline{a} b \# ab, 3) = \quad (by \; \delta(left, a) = \langle mid, \rightarrow \rangle) \\
    \delta^{31}(mid, \triangleright \sqcup a\underline{b} \# ab, 4) = \quad (by \; \delta(mid, b) = \langle mid, \rightarrow \rangle) \\
    \delta^{31}(mid, \triangleright \sqcup ab\underline{\#} ab, 5) = \quad (by \; \delta(mid, \#) = \langle right, \rightarrow \rangle) \\
    \delta^{30}(right, \triangleright \sqcup ab \# \underline{a}b, 6) = \quad (by \; \delta(right, a) = \langle seek, \rightarrow \rangle) \\
    \delta^{29}(seek, \triangleright \sqcup ab \# a\underline{b}, 7) = \quad (by \; \delta(seek, b) = \langle seek, \rightarrow \rangle) \\
    \delta^{28}(seek, \triangleright \sqcup ab \# ab\_, 8) = \quad (by \; \delta(seek, \sqcup) = \langle read, \leftarrow \rangle) \\
    \delta^{27}(read, \triangleright \sqcup ab \# a\underline{b}, 7) = \quad (by \; \delta(read, b) = \langle read_b, \$ \rangle) \\
    \delta^{26}(read_b, \triangleright \sqcup ab \# a\underline{\$}, 7) = \quad (by \; \delta(read_b, \$) = \langle read_b, \leftarrow \rangle) \\
    \delta^{25}(read_b, \triangleright \sqcup ab \# \underline{a}\$, 6) = \quad (by \; \delta(read_b, a) = \langle read_b, \leftarrow \rangle) \\
    \delta^{24}(read_b, \triangleright \sqcup ab \underline{\#} a\$, 5) = \quad (by \; \delta(read_b, \#) = \langle match_b, \leftarrow \rangle) \\
    \delta^{23}(match_b, \triangleright \sqcup a\underline{b} \# a\$, 4) = \quad (by \; \delta(match_b, b) = \langle cross, \$ \rangle) \\
    \delta^{22}(cross, \triangleright \sqcup a\underline{\$} \# a\$, 4) = \quad (by \; \delta(cross, \$) = \langle cross, \rightarrow \rangle) \\
    \delta^{21}(cross, \triangleright \sqcup a\$ \underline{\#} a\$, 5) = \quad (by \; \delta(cross, \#) = \langle scout, \rightarrow \rangle) \\
    \delta^{21}(cross, \triangleright \sqcup a\$ \underline{\#} a\$, 5) = \quad (by \; \delta(cross, \#) = \langle scout, \rightarrow \rangle) \\
    \delta^{20}(scout, \triangleright \sqcup a\$ \# \underline{a}\$, 6) = \quad (by \; \delta(scout, a) = \langle seek, \rightarrow \rangle) \\
    \delta^{19}(seek, \triangleright \sqcup a\$ \# a\underline{\$}, 7) = \quad (by \; \delta(seek, \$) = \langle read, \leftarrow \rangle) \\
    \delta^{18}(read, \triangleright \sqcup a\$ \# \underline{a}\$, 5) = \quad (by \; \delta(read, a) = \langle read_a, \$ \rangle) \\
    \delta^{17}(read_a, \triangleright \sqcup a\$ \# \underline{\$}\$, 5) = \quad (by \; \delta(read_a, \$) = \langle read_a, \leftarrow \rangle) \\
    \delta^{16}(read_a, \triangleright \sqcup a\$ \underline{\#} \$\$, 4) = \quad (by \; \delta(read_a, \#) = \langle match_a, \leftarrow \rangle) \\
    \delta^{15}(match_a, \triangleright \sqcup a \underline{\$} \# \$\$, 4) = \quad (by \; \delta(match_a, \$) = \langle match_a, \leftarrow \rangle) \\
    \delta^{14}(match_a, \triangleright \sqcup \underline{a} \$ \# \$\$, 3) = \quad (by \; \delta(match_a, a) = \langle cross, \$ \rangle) \\
    \delta^{13}(cross, \triangleright \sqcup \underline{\$} \$ \# \$\$, 3) = \quad (by \; \delta(cross, \$) = \langle cross, \rightarrow \rangle) \\
    \delta^{12}(cross, \triangleright \sqcup \$ \underline{\$} \# \$\$, 4) = \quad (by \; \delta(cross, \$) = \langle cross, \rightarrow \rangle) \\
    \delta^{11}(cross, \triangleright \sqcup \$\$ \underline{\#} \$\$, 5) = \quad (by \; \delta(cross, \#) = \langle scout, \rightarrow \rangle) \\
    \delta^{10}(scout, \triangleright \sqcup \$\$ \# \underline{\$} \$, 6) = \quad (by \; \delta(scout, \$) = \langle finish, \leftarrow \rangle) \\
    \delta^{9}(finish, \triangleright \sqcup \$\$ \underline{\#} \$\$, 5) = \quad (by \; \delta(finish, \#) = \langle finish, \leftarrow \rangle) \\
    \delta^{8}(finish, \triangleright \sqcup \$ \underline{\$} \# \$\$, 4) = \quad (by \; \delta(finish, \$) = \langle finish, \leftarrow \rangle) \\
    \delta^{7}(finish, \triangleright \sqcup \underline{\$}\$ \# \$\$, 3) = \quad (by \; \delta(finish, \$) = \langle finish, \leftarrow \rangle) \\
    \delta^{6}(finish, \triangleright \underline{\sqcup} \$\$ \# \$\$, 2) = \quad (by \; \delta(finish, \sqcup) = yes) \\
    \langle yes, \triangleright \underline{\sqcup} \$\$ \# \$\$ \rangle
\end{align*}
Следователно \(halt(ab\#ab) = yes\).

\subsection*{Задачи за упражнение}

\subsubsection*{Задача 1.}
Ако \(n = |\Sigma|\) намерете \(|Q|\).

\subsubsection*{Задача 2.}
Ако \(n = |\Sigma|\) определете процентът на полезни преходи, който се пресмята по формулата
\[\displaystyle\frac{\text{брой преходи необходими за приемане на дума}}{\text{общ брой преходи}}\]

Пресметнете приблизителната стойност за \(n = 10\). \\

Например процентът полезни преходи (в случая на генератор, такива необходими за достигане до \(h\)) за машината от \textbf{Пример 1} e \(\frac{5}{12}\),
а за машината от \textbf{Пример 2} e \(\frac{5}{8}\). Тоест приблизително \(42\%\) и \(62.5\%\).

\subsubsection*{Задача 3.}
Ако \(\alpha \in \Sigma\) и \(n = |\alpha|\) пресметнете за колко на брой стъпки построената ЕМТ приема думата \(\alpha\#\alpha\).

\subsubsection*{Задача 4.}
Нека \(M \in \mathcal{P}(\mathbb N)\) казваме, че \(M\) е разрешимо множество,
ако езикът \(B[M] = \{B(n) \mid n \in M\} \subseteq Bin\) е разрешим.
Покажете ЕМТ, която разрешава множеството \(\{2n \mid n \in \mathbb N\}\).

\subsubsection*{Задача 5.}
Покажате ЕМТ, която изчислява функцията \(f : \mathbb N \to \mathbb N\), действаща по правилото \(f(n) = 2n + 1\).

\end{document}
