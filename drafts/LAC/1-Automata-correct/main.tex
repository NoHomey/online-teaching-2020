\documentclass[12pt]{article}
\usepackage{tikz}
\usetikzlibrary{automata, positioning, arrows}
\tikzset{
->, >=stealth, node distance=3cm,
every state/.style={thick, fill=gray!10},
initial text=$ $
}
\usepackage[utf8]{inputenc}
\usepackage[T2A]{fontenc}
\usepackage[english,bulgarian]{babel}
\usepackage{amsmath}
\usepackage{amssymb}
\usepackage{amsthm}
\usepackage{relsize}
\usepackage{tensor}

\title{Лесен пример за разбиране на коректност на автомат}
\author{Иво Стратев}

\begin{document}
\maketitle

Разглеждаме един от най-простите езици
\begin{align*}
    \mathcal{L} = \{\omega \in \{a, b\}^* \; | \; \mathcal{N}_a(\omega) \equiv 0 \pmod{2}\}
\end{align*}
Един очевиден автомат за този език е
\begin{center}
\begin{tikzpicture}
    \node[state, accepting, initial] (q1) {$0$};
    \node[state, right of=q1] (q2) {$1$};
    
    \draw   (q1) edge[loop above] node{b} (q1)
            (q1) edge[bend left, above] node{a} (q2)
            (q2) edge[loop above] node{b} (q2)
            (q2) edge[bend left, below] node{a} (q1);
\end{tikzpicture}
\end{center}
Формално \(\mathcal{A} = <\{a, b\}, \{0, 1\}, 0, \delta, \{0\}>\),
където
\begin{align*}
    \delta(0, b) = 0 \\
    \delta(0, a) = 1 \\
    \delta(1, b) = 1 \\
    \delta(1, a) = 0
\end{align*}
Искаме да докажем, че автомата \(\mathcal{A}\) разпознава точно езика \(\mathcal{L}\).
Тоест \(\mathcal{L}(\mathcal{A}) = \{\omega \in \{a, b\}^* \; | \; \delta^*(0, \omega) \in \{0\}\} = \mathcal{L}\).
За тази цел формулираме по една лема за всяко състояние, с която целим да докажем
кои са езиците съотвестващи на всяко състояние. Тоест как изглеждат думите,
които стигат (привършват) до конкретно състояние.
Конкретно за разглеждания пример лемите са следните:
\begin{align*}
    (\forall \omega \in \{a, b\}^*)(\delta^*(0, \omega) = 0 \longleftrightarrow \mathcal{N}_a(\omega) \equiv 0 \pmod{2}) \\
    (\forall \omega \in \{a, b\}^*)(\delta^*(0, \omega) = 1 \longleftrightarrow \mathcal{N}_a(\omega) \equiv 1 \pmod{2})
\end{align*}
Сега ако \(q \in \{0, 1\}\) (е състояние на дадения автомат), то езика съотвестващ на това състояние ще бележим с \(\mathcal{L}_\mathcal{A}(q)\) и
\begin{align*}
\mathcal{L}_\mathcal{A}(q) = \{\omega \in \{a, b\}^* \; | \; \delta^*(0, \omega) = q\}
\end{align*}
Съответно ако означим с \(L_i = \{\omega \in \{a, b\}^* \; | \; \mathcal{N}_a(\omega) \equiv i \pmod{2}\}\) за \(i \in \{0, 1\}\). То двете твърдения, които искаме да докажем можем да запишем като
\begin{align*}
    (\forall \omega \in \{a, b\}^*)(\omega \in \mathcal{L}_\mathcal{A}(0) \longleftrightarrow \omega \in L_0) \\
    (\forall \omega \in \{a, b\}^*)(\omega \in \mathcal{L}_\mathcal{A}(1) \longleftrightarrow \omega \in L_1) 
\end{align*}
, които са еквивалетни с \(\mathcal{L}_\mathcal{A}(0) = L_0 \; \& \; \mathcal{L}_\mathcal{A}(1) = L_1\).
Вземайки предвид, че \(\mathcal{L} = L_0\) и \(\mathcal{L}(\mathcal{A}) = \mathcal{L}_\mathcal{A}(0)\). Ще получим желаното твърдение \(\mathcal{L}(\mathcal{A}) = \mathcal{L}\).
\section*{Забележка:}
В общия случай, когато автомата има повече от едно финално състояние и искаме да докажем, че \(\mathcal{L}(\mathcal{A}) = \mathcal{L}\).
То преди да докажем, че \(\mathcal{L}_\mathcal{A}(q) = L_q\) е добре да докажем, че
\(\mathcal{L} = \displaystyle\bigcup_{q \in F} L_q\), за да сме сигурни, че не изпускаме нещо. Това трябва задължително да се докаже, но ако започнем с него може да хванем някой бъг предварително ;). Равенството \(\mathcal{L}(\mathcal{A}) = \displaystyle\bigcup_{q \in F} \mathcal{L}_\mathcal{A}(q)\) е винаги в сила. Ето бързо доказателство:
\begin{align*}
\mathcal{L}(\mathcal{A}) = \{\omega \in \{a, b\}^* \; | \; \delta^*(0, \omega) \in F\} = \\
\displaystyle\bigcup_{q \in F} \{\omega \in \{a, b\}^* \; | \; \delta^*(0, \omega) = q\} =  \displaystyle\bigcup_{q \in F} \mathcal{L}_\mathcal{A}(q)
\end{align*}
Обрато към нашия пример.
Нека с \(\varphi(\alpha)\) означим свойството
\begin{align*}
    ((\alpha \in \mathcal{L}_\mathcal{A}(0) \longleftrightarrow \alpha \in L_0) \; \& \; (\alpha \in \mathcal{L}_\mathcal{A}(1) \longleftrightarrow \alpha \in L_1))    
\end{align*}.
Доказателството на \(\mathcal{L}_\mathcal{A}(0) = L_0 \; \& \; \mathcal{L}_\mathcal{A}(1) = L_1\) ще направим чрез индукция по построението на
\(\{a, b\}^*\). 
Тоест доказваме твърденията едновременно по следната схема:
\begin{enumerate}
    \item \(\varphi(\varepsilon)\)
    \item Ако \(\omega \in \{a, b\}^*\) и \(u \in \{a, b\}\) и за \(\omega\) твърдението е в сила, тоест \(\varphi(\omega)\) е истина, то показваме и че за \(\omega.u\) е в сила, тоест \(\varphi(\omega.u)\) е истина
\end{enumerate}
, с което същност доказваме
\begin{align*}
    (\forall \omega \in \{a, b\}^*)((\omega \in \mathcal{L}_\mathcal{A}(0) \longleftrightarrow \omega \in L_0) \; \& \; (\omega \in \mathcal{L}_\mathcal{A}(1) \longleftrightarrow \omega \in L_1))
\end{align*},
от където вече следва желаното.
\section*{Доказателство (стигнахме и до него ...)}
\subsection*{За \(\varepsilon\):}
По дефиниция \(\delta^*(0, \varepsilon) = 0\).
От друга страна \(\mathcal{N}_a(\varepsilon) = 0\) и \(0 \equiv 0 \pmod{2}\).
Така очевидно е вярно твърдението \(\varepsilon \in \mathcal{L}_\mathcal{A}(0) \longleftrightarrow \varepsilon \in L_0\).
От друга страна \(\delta^*(0, \varepsilon) = 1\) не е вярно и значи от това следва
\(\mathcal{N}_a(\varepsilon) \equiv 1 \pmod{2}\) (тривиално следствие).
Тоест вярно е \(\varepsilon \in \mathcal{L}_\mathcal{A}(1) \longrightarrow \varepsilon \in L_1\). По аналогични причини директно следва \(\varepsilon \in L_1 \longrightarrow \varepsilon \in \mathcal{L}_\mathcal{A}(1)\). Значи \(\varphi(\epsilon)\) е истина.
\subsection*{Стъпка:}
Нека \(\omega \in \{a, b\}^*\) и \(u \in \{a, b\}\) и за \(\omega\) твърдението е в сила,
тоест \(\varphi(\omega)\) е истина. Ще покажем \(\varphi(\omega.u)\).
Ще докажем двете твърдения в двете посоки.
Като първо ще докажем двете посоки от ляво на дясно, след това двете от дясно на ляво.
\begin{enumerate}
    \item Нека \(\delta^*(0, \omega.u) = 0\).
    Понеже \(\delta^*(0, \omega.u) = \delta(\delta^*(0, \omega), u) = 0\),
    то са възможни само два случая \(\delta^*(0, \omega) = 0 \; \& \; u = b\) или 
    \(\delta^*(0, \omega) = 1 \; \& \; u = a\). Другите два случая не са възможни, няма как да се реализират заради дефиницията на \(\delta\) функцията и \(\delta^*\).
    Ако \(\delta^*(0, \omega) = 0\) и \(u = b\), то от \(\varphi(\omega)\), имаме че \(\mathcal{N}_a(\omega) \equiv 0 \pmod{2}\), тогава очевидно \(\mathcal{N}_a(\omega.b) \equiv 0 \pmod{2}\)
    и значи \(\omega \in L_0\).
    Ако \(\delta^*(0, \omega) = 1\) и \(u = a\), то от \(\varphi(\omega)\), имаме че \(\mathcal{N}_a(\omega) \equiv 1 \pmod{2}\), тогава очевидно \(\mathcal{N}_a(\omega.a) \equiv 1 + 1 \equiv 0 \pmod{2}\)
    и значи \(\omega \in L_0\).
    Така получава, че е вярно твърдението
    \(\omega \in \mathcal{L}_\mathcal{A}(0) \longrightarrow \omega \in L_0\).
    \item Нека \(\delta^*(0, \omega.u) = 1\). Аналогично 
    понеже \(\delta^*(0, \omega.u) = \delta(\delta^*(0, \omega), u) = 1\),
    то са възможни само два случая \(\delta^*(0, \omega) = 1 \; \& \; u = b\) или 
    \(\delta^*(0, \omega) = 0 \; \& \; u = a\). Другите два случая не са възможни, няма как да се реализират заради дефиницията на \(\delta\) функцията и \(\delta^*\).
    Ако \(\delta^*(0, \omega) = 1\) и \(u = b\), то имаме, че \(\mathcal{N}_a(\omega) \equiv 1 \pmod{2}\), тогава очевидно \(\mathcal{N}_a(\omega.b) \equiv 1 \pmod{2}\)
    и значи \(\omega \in L_1\).
    Ако \(\delta^*(0, \omega) = 0\) и \(u = a\), то имаме, че \(\mathcal{N}_a(\omega) \equiv 0 \pmod{2}\), тогава очевидно \(\mathcal{N}_a(\omega.a) \equiv 0 + 1 \equiv 1 \pmod{2}\)
    и значи \(\omega \in L_1\).
    Така получава, че е вярно твърдението
    \(\omega \in \mathcal{L}_\mathcal{A}(1) \longrightarrow \omega \in L_1\).
    \item Нека \(\mathcal{N}_a(\omega.u) \equiv 0 \pmod{2}\). Тогава са възможни само два случая \(\mathcal{N}_a(\omega) \equiv 0 \pmod{2} \; \& \; u = b\) или \(\mathcal{N}_a(\omega) \equiv 1 \pmod{2} \; \& \; u = a\).
    Ако \(\mathcal{N}_a(\omega) \equiv 0 \pmod{2}\) и \(u = b\).
    Тогава понеже \(\varphi(\omega)\), то \(\delta^*(0, \omega) = 0\).
    Тогава \(\delta^*(0, \omega.u) = \delta(\delta^*(0, \omega), b) = \delta(0, b) = 0\).
    Ако \(\mathcal{N}_a(\omega) \equiv 1 \pmod{2}\) и \(u = a\).
    Тогава понеже \(\varphi(\omega)\), то \(\delta^*(0, \omega) = 1\).
    Тогава \(\delta^*(0, \omega.u) = \delta(\delta^*(0, \omega), a) = \delta(1, a) = 0\).
    Така получаваме, че е в сила \(\omega \in L_0 \longrightarrow \omega \in \mathcal{L}_\mathcal{A}(0)\).
    \item Нека \(\mathcal{N}_a(\omega.u) \equiv 1 \pmod{2}\). По аналогични на разсъжденията за предното получаваме \(\omega \in L_1 \longrightarrow \omega \in \mathcal{L}_\mathcal{A}(1)\) ...
\end{enumerate}
Така очевидно е вярно \(\varphi(\omega.u)\).
Така както вече казахме получаваме \\
\(\mathcal{L}(\mathcal{A}) = \mathcal{L}_\mathcal{A}(0) = L_0 = \mathcal{L}\).
\section*{За формалистите:}
Формално се възползваме от това, че в множеството \(\{a, b\}^*\) можем да правим индукция.
Защо ще разберете когато стигнете 3-ти курс в някой от курсовете по СЕП или Теория на множествата! Иначе правилото за индукция в случая на произволна азбука \(\Sigma\) и произволно свойство \(\varphi\) на думите над \(\Sigma\) изглежда така
\begin{align*}
    (\varphi(\varepsilon) \; \& \; (\forall \omega \in \Sigma^*)(\varphi(\omega) \longrightarrow (\forall u \in \Sigma) \varphi(\omega.u)) \longrightarrow (\forall \omega \in \Sigma^*)\varphi(\omega) 
\end{align*}
\end{document}
