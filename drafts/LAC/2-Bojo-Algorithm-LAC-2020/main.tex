\documentclass[12pt]{article}
\usepackage{tikz}
\usetikzlibrary{automata, positioning, arrows}
\tikzset{
->, >=stealth, node distance=2.5cm,
every state/.style={thick},
initial text=$ $
}
\usepackage[utf8]{inputenc}
\usepackage[T2A]{fontenc}
\usepackage[english,bulgarian]{babel}
\usepackage{amsmath}
\usepackage{amssymb}
\usepackage{amsthm}
\usepackage{relsize}
\usepackage{color}
\usepackage{upquote}
\usepackage{listings}

\definecolor{mediumgray}{rgb}{0.3, 0.4, 0.4}
\definecolor{mediumblue}{rgb}{0.0, 0.0, 0.8}
\definecolor{forestgreen}{rgb}{0.13, 0.55, 0.13}
\definecolor{darkviolet}{rgb}{0.58, 0.0, 0.83}
\definecolor{royalblue}{rgb}{0.25, 0.41, 0.88}
\definecolor{crimson}{rgb}{0.86, 0.8, 0.24}

\lstdefinelanguage{JavaScript}{
  morekeywords=[1]{break, continue, delete, else, for, function, if, in,
    new, return, this, typeof, var, void, while, with},
  morekeywords=[2]{false, null, true, boolean, number, undefined,
    Array, Boolean, Date, Math, Number, String, Object},
  morekeywords=[3]{eval, parseInt, parseFloat, escape, unescape},
  sensitive,
  morecomment=[s]{/*}{*/},
  morecomment=[l]//,
  morecomment=[s]{/**}{*/},
  morestring=[b]',
  morestring=[b]"
}[keywords, comments, strings]

\lstalias[]{ES6}[ECMAScript2015]{JavaScript}

\lstdefinelanguage[ECMAScript2015]{JavaScript}[]{JavaScript}{
  morekeywords=[1]{await, async, case, catch, class, const, default, do,
    enum, export, extends, finally, from, implements, import, instanceof,
    let, static, super, switch, throw, try},
  morestring=[b]`
}

\lstdefinestyle{JSES6Base}{
  backgroundcolor=\color{white},
  basicstyle=\ttfamily,
  breakatwhitespace=false,
  breaklines=false,
  captionpos=b,
  columns=fullflexible,
  commentstyle=\color{mediumgray}\upshape,
  emph={},
  emphstyle=\color{crimson},
  extendedchars=true,
  fontadjust=true,
  frame=single,
  identifierstyle=\color{black},
  keepspaces=true,
  keywordstyle=\color{mediumblue},
  keywordstyle={[2]\color{darkviolet}},
  keywordstyle={[3]\color{royalblue}},
  numbers=left,
  numbersep=5pt,
  numberstyle=\tiny\color{black},
  rulecolor=\color{black},
  showlines=true,
  showspaces=false,
  showstringspaces=false,
  showtabs=false,
  stringstyle=\color{forestgreen},
  tabsize=2,
  title=\lstname
}

\lstdefinestyle{JavaScript}{
  language=JavaScript,
  style=JSES6Base
}
\lstdefinestyle{ES6}{
  language=ES6,
  style=JSES6Base
}

\title{Алгоритъм на Божозовски за строене на хубав* автомат}
\author{Иво Стратев}

\begin{document}
\maketitle

\section{Да си припомним някои неща}

Със \(\Large\Sigma\) означавахме крайната азбука, над която работим. Това беше просто крайно и непразно множество.
Множеството на всички думи над азбуката \(\Large\Sigma\) бележихме със \(\Large\Sigma^*\).
Език над азбуката \(\Large\Sigma\)  за нас беше всяко подмножество на \(\Large\Sigma^*\).
Значи множеството на всички езици над азбуката \(\Large\Sigma\) е степенното множество на \(\Large\Sigma^*\),  тоест \(\Large{\mathcal{P}(\Sigma^*)}\).
Празният език е \(\emptyset\), независимо от азбуката.
Езикът на празната дума е \(\{\varepsilon\}\), независимо от азбуката.
Ако \(u\ \in \Sigma\), то \(\{u\}\) е езикът на думата с единствена буква \(u\) и \(\{u\}\) е език над \(\Sigma\).
Ако \(A\) и \(B\) са езици над \(\Sigma\), то конкатенацията на \(A\) и \(B\) беше \(A \cdot B = \{\alpha.\beta \; \mid \; \alpha \in A \; \& \; \beta \in B\} \) и също е език над същата азбука \(\Sigma\).
Ако \(A\) и \(B\) са езици над \(\Sigma\), то \(A \cup B\) също е език над \(\Sigma\). \\

\vspace{3mm}

Нека \(L\) е език над \(\Large\Sigma\). Тогава
\(L^0 = \{\varepsilon\}\) и ако \(n \in \mathbb{N}\), то \(L^{n + 1} = L \cdot L^n\) и \(L^* = \displaystyle{\bigcup_{k \in \mathbb{N}}} L^k\)
и \(L^+ = L \cdot L^* = \displaystyle{\bigcup_{k \in \mathbb{N}}} L^{k + 1}\).
Очевидно \(L^*\) и \(L^+\) са езици над \(\Sigma\).

\subsection{Примери (Сметки)}

Имаме \(\{a\}^* = \displaystyle{\bigcup_{n \in \mathbb{N}}} \{a\}^n
= \displaystyle{\bigcup_{n \in \mathbb{N}}} \{a^n\}
= \{a^n \; \mid \; n \in \mathbb{N}\}\) и \(\{a\}^+ = \{a\} \cdot \{a\}^* \). \\
Значи \(\{a\}^*
= \{a^0\} \cup \{a^n \; \mid \; n \in \mathbb{N}_+\}
= \{\varepsilon\} \cup \{a\}^+ 
= \{\varepsilon\} \cup \{a\} \cdot \{a\}^*\).

Да направим една по-голяма сметка:
\begin{align*}
L \\
= \{a, b\}^* \cdot \{c\}^*
= (\{\varepsilon\} \cup \{a, b\}^+) \cdot \{c\}^* \\
= (\{\varepsilon\} \cup (\{a, b\} \cdot \{a, b\}^* ) \cdot \{c\}^*  \\
= (\{\varepsilon\} \cup (\{a\} \cup \{b\}) \cdot \{a, b\}^* ) \cdot \{c\}^* \\
= (\{\varepsilon\} \cup \{a\} \cdot \{a, b\}^* \cup \{b\} \cdot \{a, b\}^* ) \cdot \{c\}^* \\
= \{\varepsilon\} \cdot \{c\}^* \cup \{a\} \cdot \{a, b\}^* \cdot \{c\}^* \cup \{b\} \cdot \{a, b\}^* \cdot \{c\}^* \\
= \{c\}^* \cup \{a\} \cdot L \cup \{b\} \cdot L \\
= \{\varepsilon\} \cup \{c\} \cdot \{c\}^* \cup \{a\} \cdot L \cup \{b\} \cdot L
\end{align*}
Следователно \(L\) може да бъде представен като обединение на езици започващи с конкретна буква
или на празната дума. Нещо, което ще ни бъде полезно за алгоритъма, който ще разгледаме.

\section{Операцията, която ще ни трябва}
Нека \(\Sigma\) е крайна азбука. \\
Дефинириме операция
\(remove_\Sigma \; : \; \Sigma \times \mathcal{P}(\Sigma^*)  \to \mathcal{P}(\Sigma^*)\) по следния начин\\
\(remove_\Sigma(x, L) = \{\alpha \in \Sigma^* \; \mid \; x.\alpha \in L\}\).
Очевидно тя е коректно дефинирана операция, тоест пресмятайки я върху конкретна буква \(x\) от \(\Sigma\) и конкретен език \(L\) над \(\Sigma\) получаваме 
друг език над \(\Sigma\) \(\{\alpha \in \Sigma^* \; \mid \; x.\alpha \in L\}\),
защото отделяме от \(\Sigma^*\).
Идеята зад дефинираната операция е все едно да махнем буквата \(x\) от думите започващи с \(x\) от \(L\)
и като резултат да получаваме остатъците от думите след махането.
Като под махан разбираме прочитане на буквата.

\subsection{Примерни сметки:}
Очевидно \(remove_\Sigma(x, \emptyset) = \emptyset\) понеже в празният език няма думи. \\
Също \(remove_\Sigma(x, \{\varepsilon\}) = \{\alpha \in \Sigma^* \; \mid \; x.\alpha \in \{\varepsilon\}\} = \emptyset\)
понеже \(\{\varepsilon\}\) има единствена дума, която няма букви, тоест няма какво да махнем от нея.\\
Ако \(x \in \Sigma\), то \(remove_\Sigma(x, \{x\}) = \{\alpha \in \Sigma^* \; \mid \; x.\alpha \in \{x.\varepsilon\}\} = \{\varepsilon\}\),
понеже \(\{x\} = \{x.\varepsilon\}\). \\
Ако \(x\) и \(y\) са различни букви от \(\Sigma\), то \\
\(remove_\Sigma(x, \{y\}) = \{\alpha \in \Sigma^* \; \mid \; x.\alpha \in \{y\}\} = \emptyset\). \\

\vspace{3mm}

За момента фиксираме азбуката \(\Sigma = \{a, b, c\}\) и за това вместо \(remove_{\{a, b, c\}}\) ще пишем просто \(rem\).
Нека направим няколко сметки:
\begin{align*}
    rem(a, \{aa\}) = \{a\} \quad rem(a, \{ba\}) = \emptyset \\
    rem(a, \{aa, ba\}) = rem(a, \{aa\} \cup \{ba\}) = rem(a, \{aa\}) \cup rem(a, \{ba\}) = \{a\} \cup \emptyset = \{a\} \\
    rem(a, \{a\}^+) = rem(a, \{a\} \cdot \{a\}^*) = \{a\}^* \\
    rem(a, \{a\}^*) = rem(a, \{ \varepsilon \} \cup \{a\} \cdot \{a\}^*) = \{a\}^* \\
    rem(a, \{aa\}^+) = rem(a, \{aa\} \cdot \{aa\}^*) = \{a\} \cdot \{aa\}^*
\end{align*}

Нека \(L = \{a\}^* \cdot \{b\}^+ = (\{\varepsilon\} \cup \{a\} \cdot \{a\}^*) \cdot \{b\}^+ = \{b\}^+ \cup  \{a\} \cdot \{a\}^* \cdot \{b\}^+ = \{b\} \cdot \{b\}^* \cup \{a\} \cdot L \). Тогава

\begin{align*}
    rem(a, L ) = rem(a, \{b\} \cdot \{b\}^* \cup \{a\} \cdot L) = L \\
    rem(b, L ) = rem(b, \{b\} \cdot \{b\}^* \cup \{a\} \cdot L) = \{b\}^* \\
    rem(c, L ) = rem(c, \{b\} \cdot \{b\}^* \cup \{a\} \cdot L) = \emptyset \\
    rem(a, \{b\}^*) = \emptyset \\
    rem(b, \{b\}^*) = \{b\}^* \\
    rem(a, \{b\}^*) = \emptyset \\
    rem(a, \emptyset) = rem(b, \emptyset) = rem(c, \emptyset) = \emptyset 
\end{align*}

Значи
\begin{align*}
    rem(a, L) = L \\
    rem(b, L) = \{b\}^* \\
    rem(c, L) = \emptyset \\
    rem(a, \{b\}^*) = \emptyset \\
    rem(b, \{b\}^*) = \{b\}^* \\
    rem(c, \{b\}^*) = \emptyset \\
    rem(a, \emptyset) = \emptyset \\
    rem(b, \emptyset) = \emptyset \\
    rem(c, \emptyset) = \emptyset
\end{align*}

Ако "навържем" сметките от преди малко получаваме следния автомат:

\begin{center}
\begin{tikzpicture}
    \node[state, initial] (q1) {$L$};
    \node[state, right of=q1] (q2) {$\{b\}^*$};
    \node[state, below of=q1] (q3) {$\emptyset$};
    
    \draw   (q1) edge[loop above] node{a} (q1)
            (q1) edge[bend left, above] node{b} (q2)
            (q2) edge[loop right] node{b} (q2)
            (q2) edge[bend left, below right] node{$a, c$} (q3)
            (q3) edge[loop below, below] node{$a, b, c$} (q3)
            (q1) edge[bend right, left] node{c} (q3);
\end{tikzpicture}
\end{center}

Така констуирания автомат няма нито едно финално състояние, тоест той разпознава празният език.
На нас ни се иска, той да разпознава езикът \(L = \{a\}^* \cdot \{b\}^+\).
За целта финални са всички състояния, които съдържат празната дума \(\varepsilon\).
Имаме:

\begin{align*}
    L = \{b\} \cdot \{b\}^* \cup \{a\} \cdot L \quad \text{не съдържа празната дума} \\
    \{b\}^* = \{\varepsilon\} \cup \{b\}^+ \quad \text{съдържа празната дума} \\
    \emptyset \quad \text{не съдържа празната дума}
\end{align*}

Така само \(\{b\}^*\) става финално състояние и получаваме следния автомат
\begin{center}
\begin{tikzpicture}
    \node[state, initial] (q1) {$L$};
    \node[state, accepting, right of=q1] (q2) {$\{b\}^*$};
    \node[state, below of=q1] (q3) {$\emptyset$};
    
    \draw   (q1) edge[loop above] node{a} (q1)
            (q1) edge[bend left, above] node{b} (q2)
            (q2) edge[loop right] node{b} (q2)
            (q2) edge[bend left, below right] node{$a, c$} (q3)
            (q3) edge[loop below, below] node{$a, b, c$} (q3)
            (q1) edge[bend right, left] node{c} (q3);
\end{tikzpicture}
\end{center}
Лесно се съобразява, че този автомат разпознава \(L = \{a\}^* \cdot \{b\}^+\).

\section{Алгоритъм на Божозовски}

Това, което може да забележим е, че постройхме автомат пресмятайки операцията \(rem\) върху различни езици,
започвайки от езика, за който искаме да конструираме автомат, докато получаваме нови езици с всяка буква на азбуката.
Състоянията бяха различните езици, началното, езика за който конструираме автомат, а 
финални са състоянията, които съдържат празната дума. Така готови сме да представим чрез псевдо код алгоритъма на Божозовкси. \\

\vspace{2mm}

\begin{lstlisting}[style=ES6]
function bojoAlgorithm(language) {
    const Sigma = alphabet(language);
    let languages = [language];
    let notVisited = [language];
    let transitions = [];
    let finals = [];
    while(!notVisited.isEmpty()) {
        const current = notVisited.pop();
        for(let letter in Sigma) {
            const result = rem(letter, current);
            transitions.insert([current, letter, result]);
            if(!languages.isMember(result)) {
                notVisited.push(result);
                languages.push(result);
            }
        }
        if(includesEmptyWord(current)) {
            finals.insert(current);
        }
    }
    return [Sigma, languages, transitions, language, finals];
}
\end{lstlisting}

\subsection{Няколко практически съвета}

\begin{itemize}
    \item Преди да тръгнем да смятаме най-добре е да представим езика, от който махаме букви като \textbf{обединие} на празния език или езици \textbf{започващи с конкретна буква}.
    \item След като се сметне \(rem(x, M)\) трябва \textbf{много внимателно} да се провери, дали е \textbf{нов език}.
    \item Резултата от сметката \(rem(x, M)\) е добре да \textbf{подчертаем} ако е \textbf{нов език} (такъв какъвто не сме срещали досега) и е добре да \textbf{заградим}, ако видим, че \textbf{съдържа празната дума}.
    \item Добре е поватрарящи се езици, участващи като изрази ако не са крайни и не са кратки за писане да означаваме по някакъв начин с цел съкращаване.
\end{itemize}

\section{Пример 1}
Нека постройм по разгледания алгоритъм автомат за езикът \(L = \{a, b\}^* \cdot \{c\}^*\).
В началото видяхме, че \(L = \{\varepsilon\} \cup \{c\} \cdot \{c\}^* \cup \{a\} \cdot L \cup \{b\} \cdot L\). (направихме една по-голяма сметка)
Понеже \(L\) съдържа празната дума и за нас е нов език, то следвайки съветите \(\boxed{\underline{L}}\). \\
Започваме да смятаме:

\begin{align*}
    rem(a, \; L) = rem(a, \; \{\varepsilon\} \cup \{c\} \cdot \{c\}^* \cup \{a\} \cdot L \cup \{b\} \cdot L) = L \\
    rem(b, \; L) = rem(b, \; \{\varepsilon\} \cup \{c\} \cdot \{c\}^* \cup \{a\} \cdot L \cup \{b\} \cdot L) = L \\
    rem(c, \; L) = rem(c, \; \{\varepsilon\} \cup \{c\} \cdot \{c\}^* \cup \{a\} \cdot L \cup \{b\} \cdot L) = \boxed{\underline{\{c\}^*}} \\
    rem(a, \; \{c\}^*) = \underline{\emptyset} \\
    rem(b, \; \{c\}^*) = \emptyset \\
    rem(c, \; \{c\}^*) = \{c\}^* \\
    rem(a, \; \emptyset) = \emptyset \\
    rem(b, \; \emptyset) = \emptyset \\
    rem(c, \; \emptyset) = \emptyset
\end{align*}
Така получаваме следния автомат
\begin{center}
\begin{tikzpicture}
    \node[state, initial, accepting] (q1) {$L$};
    \node[state, accepting, right of=q1] (q2) {$\{c\}^*$};
    \node[state, right of=q2] (q3) {$\emptyset$};
    
    \draw   (q1) edge[loop above] node{$a, b$} (q1)
            (q1) edge[above] node{c} (q2)
            (q2) edge[loop above, above] node{c} (q2)
            (q2) edge[above] node{$a, b$} (q3)
            (q3) edge[loop above, above] node{$a, b, c$} (q3);
\end{tikzpicture}
\end{center}

\section{Пример 2}
Нека постройм по разгледания алгоритъм автомат за един познат за нас език.
\[L = \{\omega \in \{a, b\}^* \; \mid \; \mathcal{N}_{a}(\omega) \equiv 0 \pmod{3}\}\]
Очевидно сменяме и азбуката вече \(\Sigma = \{a, b\}\), но продължаваме да пишем просто \(rem\).
Нека изразим \(L\) в \textbf{удобен за смятане вид}, тоест като \textbf{обединие на езици започващи с фиксирана буква} или празната дума.
\begin{align*}
    L = \{\varepsilon\} \cup \{\omega \in \{a, b\}^* \; \mid \; \mathcal{N}_{a}(\omega) \equiv 0 \pmod{3} \; \& \; |\omega| > 0\} \\
    = \{\varepsilon\}
\cup
\{a\} \cdot \{\omega \in \{a, b\}^* \; \mid \; \mathcal{N}_{a}(\omega) \equiv 2 \pmod{3}\}
\cup
\{b\} \cdot \{\omega \in \{a, b\}^* \; \mid \; \mathcal{N}_{a}(\omega) \equiv 0 \pmod{3}\}
\end{align*}
Да означим \(\{\omega \in \{a, b\}^* \; \mid \; \mathcal{N}_{a}(\omega) \equiv 2 \pmod{3}\}\) с \(L_2\).
Тогава \(L = \{\varepsilon\} \cup \{a\} \cdot L_2 \cup \{b\} \cdot L\).
Следвайки съветите \(\boxed{\underline{L}}\). \\

\vspace{2mm}
Пресмятаме
\begin{align*}
    rem(a, \; L) = rem(a, \; \{\varepsilon\} \cup \{a\} \cdot L_2 \cup \{b\} \cdot L) = \underline{L_2} = \{\omega \in \{a, b\}^* \; \mid \; \mathcal{N}_{a}(\omega) \equiv 2 \pmod{3}\} \\
    rem(b, \; L) = rem(b, \; \{\varepsilon\} \cup \{a\} \cdot L_2 \cup \{b\} \cdot L) = L
\end{align*}

Изразяваме \(L_2\) като обединение:

\begin{align*}
    L_2 = \{\omega \in \{a, b\}^* \; \mid \; \mathcal{N}_{a}(\omega) \equiv 2 \pmod{3}\} \\
= \{a\} \cdot \{\omega \in \{a, b\}^* \; \mid \; \mathcal{N}_{a}(\omega) \equiv 1 \pmod{3}\} 
\cup \{b\} \cdot \{\omega \in \{a, b\}^* \; \mid \; \mathcal{N}_{a}(\omega) \equiv 2 \pmod{3}\}
\end{align*}

Да означим \(\{\omega \in \{a, b\}^* \; \mid \; \mathcal{N}_{a}(\omega) \equiv 1 \pmod{3}\}\) с \(L_1\). Тогава \(L_2 = \{a\} \cdot L_1 \cup \{b\} \cdot L_2\).  \\

\vspace{2mm}
Правим още пресмятания
\begin{align*}
    rem(a, \; L_2) = rem(a, \; \{a\} \cdot L_1 \cup \{b\} \cdot L_2) = \underline{L_1} = \{\omega \in \{a, b\}^* \; \mid \; \mathcal{N}_{a}(\omega) \equiv 1 \pmod{3}\} \\
    rem(b, \; L_2) = rem(b, \; \{a\} \cdot L_1 \cup \{b\} \cdot L_2) = L_2
\end{align*}
Изразяваме \(L_1\) като обединение:
\begin{align*}
    L_1 = \{\omega \in \{a, b\}^* \; \mid \; \mathcal{N}_{a}(\omega) \equiv 1 \pmod{3}\} \\
= \{a\} \cdot \{\omega \in \{a, b\}^* \; \mid \; \mathcal{N}_{a}(\omega) \equiv 0 \pmod{3}\} 
\cup \{b\} \cdot \{\omega \in \{a, b\}^* \; \mid \; \mathcal{N}_{a}(\omega) \equiv 1 \pmod{3}\} \\
= \{a\} \cdot L \cup \{b\} \cdot L_1
\end{align*}
Правим още две пресмятания:
\begin{align*}
    rem(a, \; L_1) = rem(a, \; \{a\} \cdot L \cup \{b\} \cdot L_1) = L \\
    rem(b, \; L_1) = rem(a, \; \{a\} \cdot L \cup \{b\} \cdot L_1) = L_1
\end{align*}
Така всички сметки са:
\begin{align*}
    rem(a, L) = L_2 \\
    rem(b, L) = L \\
    rem(a, L_2) = L_1 \\
    rem(b, L_2) = L_2 \\
    rem(a, L_1) = L \\
    rem(b, L_1) = L_1
\end{align*}
В хода на пресмятанията единствено сме подчертали \(L_2\) и \(L_1\) и в началото оградихме \(L\).
С други думи само \(L\) съдържа празната дума, значи то ще единственото финално състояние.
Така автомата разпонзаващ \\
\(L = \{\omega \in \{a, b\}^* \; \mid \; \mathcal{N}_{a}(\omega) \equiv 0 \pmod{3}\}\) е

\begin{center}
\begin{tikzpicture}
    \node[state, initial, accepting] (q1) {$L$};
    \node[state, right of=q1] (q2) {$L_2$};
    \node[state, below of=q2] (q3) {$L_1$};
    
    \draw   (q1) edge[loop above] node{b} (q1)
            (q1) edge[above] node{a} (q2)
            (q2) edge[loop above] node{b} (q2)
            (q2) edge[above, right] node{a} (q3)
            (q3) edge[loop right] node{b} (q3)
            (q3) edge[below, left] node{a} (q1);
\end{tikzpicture}
\end{center}

\section{Задачи за упражнение}
Постройте автомат по алгоритъма на Божозовски за езиците:
\begin{enumerate}
    \item \(\{a\} \cdot \{a, b\}^* \cdot \{b\} \)
    \item \((\{a, b\}^+ \cdot \{a\})^*\)
    \item \(\{\omega \in \{a, b\}^* \; \mid \; \mathcal{N}_a(\omega) \equiv 0 \pmod{2} \; \& \; \mathcal{N}_b(\omega) \equiv 1 \pmod{2} \}\)
\end{enumerate}

\end{document}
