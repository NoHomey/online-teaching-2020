\documentclass[12pt]{article}
\usepackage{tikz}
\usetikzlibrary{automata, positioning, arrows}
\tikzset{
->, >=stealth, node distance=2.5cm,
every state/.style={thick},
initial text=$ $
}
\usepackage[utf8]{inputenc}
\usepackage[T2A]{fontenc}
\usepackage[english,bulgarian]{babel}
\usepackage{amsmath}
\usepackage{amssymb}
\usepackage{amsthm}
\usepackage{relsize}
\usepackage{euler}

\title{Алгоритъм за минимизация}
\author{Иво Стратев}

\begin{document}

\maketitle

Алгоритъма, който предстой да разгледаме се базира на следното наблюдение:
състоянията на един тотален КДА, които имат едни и същи езици могат да бъдат сляти в едно състояние.

\vspace*{3mm}

\par Нека \(\mathcal{A} = \langle \Sigma, Q, s, \delta, F \rangle\). Нека \(q \in Q\) тогава езикът на \(q\) ще дефинираме по обратен начин на този когато правихме коректност на автомат. Тоест като множеството от думите, които от състояние \(q\) водят до някое финално състояние.
\[\mathcal{L}ang_\mathcal{A}(q) = \{\alpha \in \Sigma^* \;\mid\; \delta(q, \alpha) \in F\}\]
Като идеята е че по този начин \(\mathcal{L}(\mathcal{A}) = \mathcal{L}ang_\mathcal{A}(s)\).

\vspace*{3mm}

\par Въвеждаме релация в множеството на състоянията на автомата, така че две състояния да са в релация ТСТК имат един и същ език.
\[p \equiv_{\mathcal{A}} q \overset{def}{\iff} \mathcal{L}ang_\mathcal{A}(q) = \mathcal{L}ang_\mathcal{A}(p)\]

Релацията \(\equiv_{\mathcal{A}}\) е релация на еквивалетност и ако тръгнем да сливаме състояния, то сливането ще бъде точно спрямо разбиването на класове на еквивалетност спрямо \(\equiv_{\mathcal{A}}\).

\vspace*{3mm}

\par Както знаем от курса по ДС релациите на еквивалетност  над едно множество са в съотвествие с начините за разбиване на това множество. Е алгоритъма се състой в апроксимирането (приближаването) на релацията \(\equiv_{\mathcal{A}}\) чрез разбивания на множеството от състоянията. Тоест чрез апроксимиране на разбиването съотвестващо на релацията \(\equiv_{\mathcal{A}}\).

Като за апроксимацията ще са ни необходими най-много \(|Q|\) стъпки, защото в най-лошият случай всеки клас на еквивалетност е едноелементно множество.

\vspace*{3mm}

\par Започваме само с два класа на еквивалетност: \(F\) и \(Q \setminus F\).
След, което постепенно опитваме да разбием всеки клас на еквивалетност.
Като ако на някоя стъпка получим едно елементно множество, то разбира се не пробваме да го разбием.

Разбиването става спрямо преходите на автомата. Ако за едно множества от състояния има буква от азбуката, с която две състояния отиват в различен клас на еквивалетност от предната стъпка, то се разбива. Ако дори само един клас бъде разбит, то на следващата стъпка пробваме да разбием всеки от получените. Независимо, че на предната стъпка, някое множество може да не се е разбивало, това не значи, че новата стъпка гарантирано няма да се разбие.

\vspace*{3mm}

\par Така в началото започваме от една груба релация, зададена чрез класовете си на еквивалетност, които са два: финални и нефинални състояния и постепенно пробваме да изтъняваме релацията, като в резултат получаваме евентуално по-финно разбиване. Спираме когато на някоя стъпка нито един клас на еквивалетност не може да бъде разбит, това разбиване задава релацията, която търсим.

\vspace*{3mm}

\par Нека минимизираме следният креан \textbf{тотален} детерминиран автомат:

\begin{center}
\begin{tikzpicture}
    \node[state, initial] (q0) {$0$};
    \node[state, right of=q0] (q1) {$1$};
    \node[state, accepting, right of=q1] (q2) {$2$};
    \node[state, accepting, right of=q2] (q3) {$3$};
    \node[state, below of=q2] (q4) {$4$};
    \draw (q0) edge[above] node{a} (q1);
    \draw (q1) edge[loop above] node{a} (q1);
    \draw (q2) edge[above] node{b} (q3);
    \draw (q3) edge[loop above] node{b} (q3);
    \draw (q1) edge[bend left, above] node{b} (q2);
    \draw (q0) edge[bend right, below] node{b} (q2);
    \draw (q2) edge[left] node{a} (q4);
    \draw (q3) edge[below right] node{a} (q4);
    \draw (q4) edge[loop below] node{$a, b$} (q4);
\end{tikzpicture}
\end{center}

Нека \(A_0 = \{0, 1, 4\}\) и \(A_1 = \{2, 3\}\). Тогава началното разбиване е \(\{A_0, A_1\}\).
Пробваме дали можем да разбием всяко от двете множества.
\\
\vspace*{5mm}
\\
\begin{minipage}{.5\textwidth}
\centering
\(\begin{array}{c|cc|c}
A_0 & a & b   &     \\ \hline
0 & A_0 & A_1 & B_0 \\
1 & A_0 & A_1 & B_0 \\
4 & A_0 & A_0 & B_1 \\ \hline
  & \checkmark & & 
\end{array}\)
\end{minipage}
\begin{minipage}{.5\textwidth}
\centering
\(\begin{array}{c|cc|c}
A_1 & a & b   &     \\ \hline
2 & A_0 & A_1 & B_2 \\
3 & A_0 & A_1 & B_2 \\ \hline
  & \checkmark & \checkmark & 
\end{array}\)
\end{minipage}
\\
\vspace*{5mm}
\\
Чавки има в колоните, в които имаме пълно съвпадение. Както се вижда в таблицата относно \(A_0\) в колоната за буквата \(b\) има разлика и значи множеството се разбива. Разбиването е на множества спрямо редовете на таблицата. Тоест еднакивте редове попадат в един клас на екивалетност.
\vspace*{3mm}

\par Така получаваме три класа на еквивалетност \(B_0 = \{0, 1\}\), \(B_1 = \{4\}\) и \(B_2 = \{2, 3\}\).
Ясно е, че \(B_1\) няма как да бъде разбито понеже релацията, която апроксимираме е рефлексивна.
Пробваме да разбием другите две.
\\
\vspace*{5mm}
\\
\begin{minipage}{.5\textwidth}
\centering
\(\begin{array}{c|cc|c}
B_0 & a & b   & \\ \hline
0 & B_0 & B_2 & \\
1 & B_0 & B_2 & \\ \hline
  & \checkmark & \checkmark & 
\end{array}\)
\end{minipage}
\begin{minipage}{.5\textwidth}
\centering
\(\begin{array}{c|cc|c}
\hat{\delta} & a & b & kind \\ \hline
B_0 & B_0 & B_2 & start \\
B_1 & B_1 & B_1 & error \\
B_2 & B_1 & B_2 & final
\end{array}\)
\end{minipage}
\\
\vspace*{5mm}
\\
Получаваме пълни съвпадения във всяка колона. Следователно нито едно множество не може да бъде разбито.
Така разбиването \(\{B_0, B_1, B_2\}\) задава релацията, която търсим. Всеки един клас на екивалетност ще бъде състояние в минималният автомат. За това е удобно да си направим таблица, на базата на която да построим минимален автомат. Таблицата същност ще описва функцията на преходите и вида на всяко състояние.
\[\begin{array}{c|cc|c}
\hat{\delta} & a & b & kind \\ \hline
B_0 & B_0 & B_2 & start \\
B_1 & B_1 & B_1 & error \\
B_2 & B_1 & B_2 & final
\end{array}\]

Така получаваме автомата
\begin{center}
\begin{tikzpicture}
    \node[state, initial] (q0) {$\{0, 1\}$};
    \node[state, accepting, right of=q0] (q1) {$\{2, 3\}$};
    \node[state, right of=q1] (q2) {$\{4\}$};
    \draw (q0) edge[loop above] node{a} (q0);
    \draw (q0) edge[above] node{b} (q1);
    \draw (q1) edge[loop above] node{b} (q1);
    \draw (q1) edge[above] node{a} (q2);
    \draw (q2) edge[loop above] node{$a, b$} (q2);
\end{tikzpicture}
\end{center}

Полученият автомат е изоморфен на автомата
\begin{center}
\begin{tikzpicture}
    \node[state, initial] (q0) {$0$};
    \node[state, accepting, right of=q0] (q1) {$1$};
    \node[state, right of=q1] (q2) {$2$};
    \draw (q0) edge[loop above] node{a} (q0);
    \draw (q0) edge[above] node{b} (q1);
    \draw (q1) edge[loop above] node{b} (q1);
    \draw (q1) edge[above] node{a} (q2);
    \draw (q2) edge[loop above] node{$a, b$} (q2);
\end{tikzpicture}
\end{center}

Вече лесно се съобразява че езкът на двата минилимални автомата и първоначалният е \(\{a\}^* \cdot \{b\}^+\). Това не би трябвало да бъде изненада, защото това е тотален автомат на базата автомат за същия език получен чрез основните конструкции за строене на недетерминиран автомат по регулярен език. В конкретният случай конструираният автомат на предходно упражнение е детерминиран, но не е тотален.

\vspace*{3mm}

\par Нека минимизираме следният автомат

\begin{center}
\begin{tikzpicture}
    \node[state, initial] (q0) {$0$};
    \node[state, right of=q0] (q1) {$1$};
    \node[state, accepting, below of=q1] (q2) {$2$};
    \node[state, accepting, below right of=q1] (q3) {$3$};
     \node[state, below of=q2] (q4) {$4$};
    \draw (q0) edge[above] node{a} (q1);
    \draw (q1) edge[right] node{b} (q2);
    \draw (q1) edge[above right] node{c} (q3);
    \draw (q2) edge[right] node{$b, c$} (q4);
    \draw (q3) edge[loop right] node{a} (q3);
    \draw (q4) edge[loop right] node{$a, b, c$} (q4);
    \draw (q3) edge[bend left, right] node{$b, c$} (q4);
    \draw (q0) edge[bend right, left] node{$b, c$} (q4);
    \draw (q1) edge[bend right, left] node{a} (q4);
    \draw (q2) edge[loop right] node{a} (q2);
\end{tikzpicture}
\end{center}

Започваме от разбиванетo \(\{A_0, A_1\}\), където \(A_0 = \{0, 1, 4\}\) и \(A_1 = \{2, 3\}\).
Пробваме да разбием всяко едно от двете множества.
\\
\vspace*{5mm}
\\
\begin{minipage}{.5\textwidth}
\centering
\(\begin{array}{c|ccc|c}
A_0 & a & b   & c     &     \\ \hline
0   & A_0 & A_0 & A_0 & B_0 \\
1   & A_0 & A_1 & A_1 & B_1 \\
4   & A_0 & A_0 & A_0 & B_0 \\ \hline
    & \checkmark & &  &
\end{array}\)
\end{minipage}
\begin{minipage}{.5\textwidth}
\centering
\(\begin{array}{c|ccc|c}
A_1 & a & b   & c     &     \\ \hline
2   & A_1 & A_0 & A_0 & B_2 \\
3   & A_1 & A_0 & A_0 & B_2 \\ \hline
    & \checkmark & \checkmark & \checkmark &
\end{array}\)
\end{minipage}
\\
\vspace*{5mm}
\\
\(A_0\) се разбива на две.

Нека \(B_0 = \{0, 4\}\), \(B_1 = \{1\}\) и \(B_2 = \{2, 3\}\).
Пробваме да разбием \(B_0\) и \(B_2\).
\\
\vspace*{5mm}
\\
\begin{minipage}{.5\textwidth}
\centering
\(\begin{array}{c|ccc|c}
B_0 & a & b   & c     &     \\ \hline
0   & B_1 & B_0 & B_0 & C_0 \\
4   & B_0 & B_0 & B_0 & C_1 \\ \hline
    &     & \checkmark & \checkmark &
\end{array}\)
\end{minipage}
\begin{minipage}{.5\textwidth}
\centering
\(\begin{array}{c|ccc|c}
B_2 & a & b   & c     &     \\ \hline
2   & B_1 & B_0 & B_0 & C_2 \\
3   & B_1 & B_0 & B_0 & C_2 \\ \hline
    & \checkmark & \checkmark & \checkmark &
\end{array}\)
\end{minipage}
\\
\vspace*{5mm}
\\
\(B_0\) се разбива на две. Нека \(C_0 = \{0\}\), \(C_1 = \{4\}\), \(C_2 = \{2, 3\}\) и \(C_3 = \{1\}\).
Пробваме да разбием само \(C_2\) понеже другите са едноелементни.

\[\begin{array}{c|ccc|c}
C_2 & a & b   & c     & \\ \hline
2   & C_2 & C_1 & C_1 & \\
3   & C_2 & C_1 & C_1 & \\ \hline
    & \checkmark & \checkmark & \checkmark &
\end{array}\]

\(C_2\) не се разбива така, че получаваме разбиването \(\{C_0, C_1, C_2, C_3\}\),
което задава релацията спрямо, която сливаме състоянията.

\vspace*{3mm}

\par Правим таблица за функцията на преходите на минимизирания автомат.
\[\begin{array}{c|ccc|c}
\hat{\delta} & a   & b   & c   & kind  \\ \hline
C_0          & C_3 & C_1 & C_1 & start \\
C_1          & C_1 & C_1 & C_1 & error \\ 
C_2          & C_2 & C_1 & C_1 & final \\ 
C_3          & C_1 & C_2 & C_2 &       \\ 
\end{array}\]

По нея строим (дигарамата на) минимизирания автомат.

\begin{center}
\begin{tikzpicture}
    \node[state, initial] (q0) {$\{0\}$};
    \node[state, right of=q0] (q1) {$\{1\}$};
    \node[state, accepting, right of=q1] (q2) {$\{2, 3\}$};
    \node[state, below of=q1] (q3) {$\{4\}$};
    \draw (q0) edge[above] node{a} (q1);
    \draw (q1) edge[above] node{$b, c$} (q2);
    \draw (q2) edge[loop right] node{a} (q2);
    \draw (q0) edge[below left] node{$b, c$} (q3);
    \draw (q1) edge[left] node{a} (q3);
    \draw (q2) edge[below right] node{$b, c$} (q3);
    \draw (q3) edge[loop right] node{$a, b, c$} (q3);
\end{tikzpicture}
\end{center}

Изоморфен на него е следния автомат:

\begin{center}
\begin{tikzpicture}
    \node[state, initial] (q0) {$0$};
    \node[state, right of=q0] (q1) {$1$};
    \node[state, accepting, right of=q1] (q2) {$2$};
    \node[state, below of=q1] (q3) {$3$};
    \draw (q0) edge[above] node{a} (q1);
    \draw (q1) edge[above] node{$b, c$} (q2);
    \draw (q2) edge[loop right] node{a} (q2);
    \draw (q0) edge[below left] node{$b, c$} (q3);
    \draw (q1) edge[left] node{a} (q3);
    \draw (q2) edge[below right] node{$b, c$} (q3);
    \draw (q3) edge[loop right] node{$a, b, c$} (q3);
\end{tikzpicture}
\end{center}

Този път езикът е \(\{a\} \cdot \{b, c\} \cdot \{a\}^*\).

\end{document}
