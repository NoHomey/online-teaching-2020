\documentclass[12pt]{article}
\usepackage[utf8]{inputenc}

\usepackage[T2A]{fontenc}
\usepackage[english,bulgarian]{babel}
\usepackage{amsmath}
\usepackage{amssymb}
\usepackage{amsthm}
\usepackage{relsize}
\usepackage{tensor}

\title{Задачи за упражнение върху (не)Определимост}
\author{Иво Стратев}

\begin{document}
\maketitle

\section{Любима задача на Иво}
Нека \(\mathcal{N}\) е структура с универсум \(\mathcal{P}(\mathbb{N})\) (степенното множество на \(\mathbb{N}\)) за език с единствен двуместен претикатен символ \(p\). \\
Където \(<A, B> \; \in p^{\mathcal{N}} \longleftrightarrow A \subseteq B\).

Да се докаже, че са определими множествата:
\begin{itemize}
    \item \(\{\emptyset\}\)
    \item \(\{\mathbb{N}\}\)
    \item \(\emptyset\)
    \item \(\mathcal{P}(\mathbb{N})\)
    \item \(\{<A, A> \; | \; A \in \mathcal{P}(\mathbb{N})\}\)
    \item \(\{<A, B, C> \; \in \mathcal{P}(\mathbb{N})^3 \; | \; C = A \cup B \}\)
    \item \(\{<A, B, C> \; \in \mathcal{P}(\mathbb{N})^3 \; | \; C = A \cap B \}\)
    \item \(\{<A, B> \; \in \mathcal{P}(\mathbb{N})^2 \; | \; B = \mathbb{N} \setminus A \}\)
    \item \(\{<A, B, C> \; \in \mathcal{P}(\mathbb{N})^3 \; | \; C = A \setminus B \}\)
    \item \(\{ \{n\} \; | \; n \in \mathbb{N}\} \)
    \item \(\{ \{n, m\} \; | \; n \in \mathbb{N} \; \& \; m \in \mathbb{N} \; \& \; n \neq m\} \)
\end{itemize}

Поглеждайки само домейна на структурата може ли да кажем, че има неопределими елементи ? Защо ?

Вярно ли е, че неопределимите елементи са повече от определимите елементи ? В какъв смисъл повече ?

Докажете, че единствените определими елементи са \(\emptyset\) и \(\mathbb{N}\).

\section{Хубава задача за усещане що е то Автоморфизъм}
На първото упражнение разгледахме следната структура 
\(\mathcal{N} = \; <\mathbb{N}, f^{\mathcal{N}}, g^{\mathcal{N}}>\)
за език с формално равенство и два функционални символа \(f\) (двуместен) и \(g\) (едноместен). Където
\begin{align*}
    f^{\mathcal{N}}(a, b) = a ._{\small\mathbb{N}} b \\
    g^{\mathcal{N}}(a) = mod(a, 7)
\end{align*}

Първо докажете, че има два неопределими елемента, след това докажете, че има изброимо (безкрайно) много неопределими елементи. Като подсказка забележете, че тази структура е  някак по-богата от структурата на естенствените числа разгледани само с умножението им (и равенство разбира се). 

\end{document}
